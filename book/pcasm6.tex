% -*-latex-*-
\chapter{Arrays}
\index{arrays|(}
\section{Introduction}

An \emph{array} is a contiguous block of data in memory. Each element
of the list must be the same type and use exactly the same number of bytes
of memory for storage. Because of these properties, arrays allow efficient
access of the data by its position (or index) in the array. The address
of any element can be computed by knowing three facts:
\begin{itemize}
\item The address of the first element of the array.
\item The number of bytes in each element
\item The index of the element
\end{itemize}

It is convenient to consider the index of the first element of the array
to be zero (just as in C). It is possible to use other values for the
first index, but it complicates the computations.

\subsection{Defining arrays\index{arrays!defining|(}}

\begin{figure}[t]
%tested in array1.asm
\begin{lstlisting}[language={[x86masm]Assembler}]
.data
	; define array of 10 double words initialized to 1,2,..,10
	a1 dd 1, 2, 3, 4, 5, 6, 7, 8, 9, 10
	; define array of 10 words initialized to 0
	a2 dw  0, 0, 0, 0, 0, 0, 0, 0, 0, 0
	; same as before using dup
	a3 dw 10 dup (0)
	; define array of bytes with 200 0's and then 100 1's
	a4 db 200 dup (0)
	   db 100 dup (1)
	; define an array of 10 uninitialized double words
	a5 dd 10 dup (?)
	; define an array of 100 uninitialized words
	a6 dw 100 dup (?)
\end{lstlisting}
\caption{Defining arrays\label{fig:DataArrays}}
\end{figure}

\subsubsection{Defining arrays in the {\code data} segment
               \index{arrays!defining!static}}

To define an initialized array in the {\code data} segment, use the
normal {\code db}, {\code dw}, \emph{etc.} 
\index{directive!D\emph{X}}directives. MASM also provides a useful directive
named {\code DUP} \index{directive!DUP} that can be used to repeat a statement many times
without having to duplicate the statements by hand.
Figure~\ref{fig:DataArrays} shows several examples of these.

To define an uninitialized array, use the
{\code ?} \index{directive!?}
directive. Figure~\ref{fig:DataArrays} also shows examples of these
types of definitions.

\begin{figure}[t]
\centering
\begin{tabular}{l|c|ll|c|}
\cline{2-2} \cline{5-5}
EBP - 1  & char    & \hspace{2em} &           & \\
\cline{2-2}
         & unused  &              &           & \\
\cline{2-2}
EBP - 8  & dword 1 &              &           & \\
\cline{2-2}
EBP - 12 & dword 2 &              &           & word \\
\cline{2-2}
         &         &              &           & array \\
         &         &              &           & \\
         & word    &              &           & \\
         & array   &              & EBP - 100 & \\
\cline{5-5}
         &         &              & EBP - 104 & dword 1 \\
\cline{5-5}
         &         &              & EBP - 108 & dword 2 \\
\cline{5-5}
         &         &              & EBP - 109 & char \\
\cline{5-5}
EBP - 112 &        &              &           & unused \\
\cline{2-2} \cline{5-5}
\end{tabular}
\caption{Arrangements of the stack\label{fig:StackLayouts}}
\end{figure}

\subsubsection{Defining arrays as local variables on the stack\index{arrays!defining!local variable}}

There is no direct way to define a local array variable on the
stack. As before, one computes the total bytes required by \emph{all}
local variables, including arrays, and subtracts this from ESP (either
directly or using the {\code ENTER} instruction). For example, if a
function needed a character variable, two double word integers and a
50 element word array, one would need $1 + 2 \times 4 + 50 \times 2 =
109$ bytes. However, the number subtracted from ESP should be a
multiple of four (112 in this case) to keep ESP on a double word
boundary. One could arrange the variables inside this 109 bytes in
several ways. Figure~\ref{fig:StackLayouts} shows two possible ways. The
unused part of the first ordering is there to keep the double words on
double word boundaries to speed up memory accesses.
\index{arrays!defining|)}

\subsection{Accessing elements of arrays\index{arrays!accessing|(}}

The {\code [ ]} operator in assembly language is much more versatile than C.  You can
use it similarly to how you might use it in C to access an array, but to
access the correct element of an array, its address must be computed. Consider
the following two array definitions:
%tested in array2.asm
\begin{lstlisting}[language={[x86masm]Assembler}]
array1  db 5, 4, 3, 2, 1     ; array of bytes
array2  dw 5, 4, 3, 2, 1     ; array of words
array3  dd 5, 4, 3, 2, 1     ; array of double words
\end{lstlisting}
Here are some examples using these arrays:
\begin{lstlisting}[language={[x86masm]Assembler}]
	mov    al, [array1]             ; al = array1[0]
	mov    al, array1[1]         ; al = array1[1]
	mov    [array1 + 3], al         ; array1[3] = al
	mov    ax, array2             ; ax = array2[0]
	mov    ax, array2[2]         ; ax = array2[1] (NOT array2[2]!)
	mov    [array2 + 6], ax         ; array2[3] = ax
	mov    ax, [array2 + 1]         ; ax = ??
\end{lstlisting}
In line~5, element 1 of the word array is referenced, not element 2. Why?
Words are two byte units, so to move to the next element of a word array,
one must move two bytes ahead, not one. Line~7 will read one byte from the
first element and one from the second. In C, the compiler looks at the type
of a pointer in determining how many bytes to move in an expression that
uses pointer arithmetic so that the programmer does not have to. However,
in assembly, it is up to the programmer to take the size of array elements
in account when moving from element to element. \index{ArrayOps(}Luckily, there is a way 
to calculate the important values associated with an array, see Table~\ref{tab:ArrayOps}.

\begin{table}[]
\begin{tabular}{ll}
\textbf{Command} & \textbf{Purpose}                    \\
type             & The number of bytes in each element \\
lengthof           & The number of elements in an array  \\
sizeof         & The number of bytes in an array    
\end{tabular}
\caption{Helpful Array Operators\label{tab:ArrayOps}}
\end{table}
\index{ArrayOps)}
%tested in array3.asm way 1
\begin{figure}[t]
\begin{lstlisting}[language={[x86masm]Assembler},frame=single]
	mov    ebx, offset array1           ; ebx = address of array1
	mov    dx, 0                 ; dx will hold sum
	mov    ah, 0                 ; ?
	mov    ecx, 5
lp:
	mov    al, [ebx]             ; al = *ebx
	add    dx, ax                ; dx += ax (not al!)
	inc    ebx                   ; bx++
	loop   lp
\end{lstlisting}
\caption{Summing elements of an array (Version 1)\label{fig:SumArray1}}
\end{figure}
%tested in array3.asm way 2
\begin{figure}[t]
\begin{lstlisting}[language={[x86masm]Assembler},frame=single]
      mov    ebx, offset array1           ; ebx = address of array1
      mov    dx, 0                 ; dx will hold sum
      mov    ecx, 5
lp:
      add    dl, [ebx]             ; dl += *ebx
      jnc    next                  ; if no carry goto next
      inc    dh                    ; inc dh
next:
      inc    ebx                   ; bx++
      loop   lp
\end{lstlisting}
\caption{Summing elements of an array (Version 2)\label{fig:SumArray2}}
\end{figure}
%tested in array3.asm way 3
\begin{figure}[t]
\begin{lstlisting}[language={[x86masm]Assembler},frame=single]
      mov    ebx, offset array1           ; ebx = address of array1
      mov    dx, 0                 ; dx will hold sum
      mov    ecx, 5
lp:
      add    dl, [ebx]             ; dl += *ebx
      adc    dh, 0                 ; dh += carry flag + 0
      inc    ebx                   ; bx++
      loop   lp
\end{lstlisting}
\caption{Summing elements of an array (Version 3)\label{fig:SumArray3}}
\end{figure}

Figure~\ref{fig:SumArray1} shows a code snippet that adds all the
elements of {\code array1} in the previous example code. In
line 7, AX is added to DX. Why not AL? First, the
two operands of the {\code ADD} instruction must be the same
size. Secondly, it would be easy to add up bytes and get a sum that
was too big to fit into a byte. By using DX, sums up to 65,535 are
allowed. However, it is important to realize that AH is being added
also.  This is why AH is set to zero\footnote{Setting AH to zero is
implicitly assuming that AL is an unsigned number. If it is signed,
the appropriate action would be to insert a {\code CBW} instruction
between lines~6 and 7} in line~3.

Figures~\ref{fig:SumArray2} and \ref{fig:SumArray3} show two alternative
ways to calculate the sum. The differences are mostly around lines~6 and 7
of each Figure.

\subsection{More advanced indirect addressing\index{indirect addressing!arrays|(}}

Not surprisingly, indirect addressing is often used with arrays. The most
general form of an indirect memory reference is:
\begin{center}
{\code [ \emph{base reg} + \emph{factor}*\emph{index reg} + 
      \emph{constant}]}
\end{center}
where:
\begin{description}
\item[base reg] is one of the registers EAX, EBX, ECX, EDX, EBP, ESP, ESI
                or EDI.
\item[factor] is either 1, 2, 4 or 8 . (If 1, factor is omitted.)
\item[index reg] is one of the registers EAX, EBX, ECX, EDX, EBP, ESI, EDI.
                 (Note that ESP is not in list.)
\item[constant] is a 32-bit constant. The constant can be a label (or
                a label expression).
\end{description}

%\subsection{Example}
Figure~\ref{fig:SumArray4} is another sum array example that uses the general form. 

%tested in array4.asm
\begin{figure}[t]
\begin{lstlisting}[language={[x86masm]Assembler},frame=single]
	mov    ebx, offset array3     ; ebx = address of array3
	mov    edx, 0                 ; edx will hold sum
	mov    ecx, 5                 ; ecx is our loop counter
lp:            ;[baseReg + factor*indexReg + constant]
	add    edx, [ebx+(type array3)*ecx-(type array3)]; edx += array3[ecx-1]
	loop   lp
\end{lstlisting}
\caption{Summing elements of an array (Version 4)\label{fig:SumArray4}}
\end{figure}

\index{indirect addressing!arrays|)}
\index{arrays!accessing|)}

\subsubsection{The {\code LEA} instruction\index{LEA|(}}

The {\code LEA} instruction is used to calculate addresses instead of 
{\code offset}. In addition to this, you could use it for other purposes. A fairly
common one is for fast computations. Consider the following:
%tested in array4.asm
\begin{lstlisting}[language={[x86masm]Assembler},numbers=none]
      lea    ebx, [4*eax + eax]
\end{lstlisting}
This effectively stores the value of $5 \times \mathtt{EAX}$ into EBX.
Using {\code LEA} to do this is both easier and faster than using
{\code MUL}\index{MUL}. However, one must realize that the expression
inside the square brackets \emph{must} be a legal indirect address.
Thus, for example, this instruction can not be used to multiple by 6
quickly.
\index{LEA|)}


\subsection{Multidimensional Arrays\index{arrays!multidimensional|(}}

Multidimensional arrays are not really very different than the plain
one dimensional arrays already discussed. In fact, they are represented 
in memory as just that, a plain one dimensional array.

\subsubsection{Two Dimensional Arrays\index{arrays!multidimensional!two dimensional|(}}
Not surprisingly, the simplest multidimensional array is a two dimensional
one. A two dimensional array is often displayed as a grid of elements. Each
element is identified by a pair of indices. By convention, the first index
is identified with the row of the element and the second index the column.

Consider an array with three rows and two columns defined as: 
\begin{lstlisting}[stepnumber=0]{}
  int a[3][2];
\end{lstlisting}
The C compiler would reserve room for a 6 ($= 2 \times 3$) integer array and
map the elements as follows:

\parbox{\textwidth}{
\vspace{0.5em}
\centering
\begin{tabular}{||l|c|c|c|c|c|c||}
\hline
Index & 0 & 1 & 2 & 3 & 4 & 5 \\
\hline
Element & a[0][0] & a[0][1] & a[1][0] & a[1][1] & a[2][0] & a[2][1]  \\
\hline
\end{tabular}
\vspace{0.5em}
}
\noindent What the table attempts to show is that the element referenced as 
{\code a[0][0]} is stored at the beginning of the 6 element one
dimensional array. Element {\code a[0][1]} is stored in the next
position (index~1) and so on. Each row of the two dimensional array is
stored contiguously in memory. The last element of a row is followed
by the first element of the next row. This is known as the
\emph{rowwise} representation of the array and is how a C/C++ compiler would
represent the array.

How does the compiler determine where {\code a[i][j]} appears in the rowwise
representation? A simple formula will compute the index from {\code i} and
{\code j}. The formula in this case is $2i + j$. It's not too hard to see how
this formula is derived. Each row is two elements long; so, the first element
of row $i$ is at position $2i$. Then the position of column $j$ is found by
adding $j$ to $2i$. This analysis also shows how the formula is generalized 
to an array with {\code N} columns: $N \times i + j$. Notice that the formula
does \emph{not} depend on the number of rows.

As an example, let us see how \emph{gcc} compiles the following code (using the
array {\code a} defined above):
\begin{lstlisting}[stepnumber=0]{}
  x = a[i][j];
\end{lstlisting}
The compiler essentially converts the code to:
\begin{lstlisting}[stepnumber=0]{}
  x = *(&a[0][0] + 2*i + j);
\end{lstlisting}
and in fact, the programmer could write this way with the same result.  The example code 
below shows how you might find the sum of a 2D array in assembly.

\lstinputlisting{../code/ch6/2dsum.asm}

There is nothing magical about the choice of the rowwise representation of the
array. A columnwise representation would work just as well: 

\parbox{\textwidth}{
\vspace{0.5em}
\centering
\begin{tabular}{||l|c|c|c|c|c|c||}
\hline
Index & 0 & 1 & 2 & 3 & 4 & 5 \\
\hline
Element & a[0][0] & a[1][0] & a[2][0] & a[0][1] & a[1][1] & a[2][1]  \\
\hline
\end{tabular}
\vspace{0.5em}
}
\noindent In the columnwise representation, each column is stored contiguously. 
Element {\code [i][j]} is stored at position $i + 3j$. Other languages
(FORTRAN, for example) use the columnwise representation. This is
important when interfacing code with multiple languages.
\index{arrays!multidimensional!two dimensional|)}

\subsubsection{Dimensions Above Two}
For dimensions above two, the same basic idea is applied. Consider a three
dimensional array:
\begin{lstlisting}[stepnumber=0]{}
  int b[4][3][2];
\end{lstlisting}
This array would be stored like it was four two dimensional arrays each of size
{\code [3][2]} consecutively in memory. The table below shows how it starts out:

\parbox{\textwidth}{
\vspace{0.5em}
\centering
\begin{tabular}{||l|c|c|c|c|c|c||}
\hline
Index & 0 & 1 & 2 & 3 & 4 & 5  \\
\hline
Element & b[0][0][0] & b[0][0][1]  & b[0][1][0] & b[0][1][1] & b[0][2][0]
&  b[0][2][1]  \\
\hline
\hline
Index & 6 & 7 & 8 & 9 & 10 & 11 \\
\hline
Element & b[1][0][0] & b[1][0][1] & b[1][1][0] & b[1][1][1]  & b[1][2][0] 
& b[1][2][1] \\
\hline
\end{tabular}
\vspace{0.5em}
}
\noindent The formula for computing the position of {\code b[i][j][k]}
is $6i + 2j + k$. The 6 is determined by the size of the {\code
[3][2]} arrays. In general, for an array dimensioned as {\code
a[L][M][N]} the position of element {\code a[i][j][k]} will be $M\times N\times i 
+ N \times j + k$. Notice again that the first
dimension ({\code L}) does not appear in the formula.

For higher dimensions, the same process is generalized. For an $n$ dimensional
array of dimensions $D_1$ to $D_n$, the position of element denoted by the
indices $i_1$ to $i_n$ is given by the formula:
\begin{displaymath}
D_2 \times D_3 \cdots \times D_n \times i_1 + D_3 \times D_4 \cdots \times D_n 
\times i_2 + \cdots + D_n \times i_{n-1} + i_n
\end{displaymath}
or for the \"{u}ber math geek, it can be written more succinctly as:
\begin{displaymath}
\sum_{j=1}^{n} \: \left( \prod_{k=j+1}^{n} D_k \right) \: i_j
\end{displaymath}
\MarginNote{This is where you can tell the author was a physics major. (Or was the
reference to FORTRAN a giveaway?)}
The first dimension, $D_1$, does not appear in the formula.

For the columnwise representation, the general formula would be:
\begin{displaymath}
i_1 + D_1 \times i_2 + \cdots + D_1 \times D_2 \times \cdots \times D_{n-2} 
\times i_{n-1} + D_1 \times D_2 \times \cdots \times D_{n-1} \times i_n
\end{displaymath}
or in \"{u}ber math geek notation:
\begin{displaymath}
\sum_{j=1}^{n} \: \left( \prod_{k=1}^{j-1} D_k \right) \: i_j
\end{displaymath}
In this case, it is the last dimension, $D_n$, that does not appear in the
formula.
%
%\subsubsection{Passing Multidimensional Arrays as Parameters in C\index{arrays!multidimensional!parameters|(}}
%
%The rowwise representation of multidimensional arrays has a direct
%effect in C programming. For one dimensional arrays, the size of the
%array is not required to compute where any specific element is located
%in memory. This is not true for multidimensional arrays.  To access
%the elements of these arrays, the compiler must know all but the first
%dimension. This becomes apparent when considering the prototype of a
%function that takes a multidimensional array as a parameter. The
%following will not compile:
%\begin{lstlisting}[stepnumber=0]{}
%  void f( int a[ ][ ] );  /* no dimension information */
%\end{lstlisting}
%However, the following does compile:
%\begin{lstlisting}[stepnumber=0]{}
%  void f( int a[ ][2] );
%\end{lstlisting}
%Any two dimensional array with two columns can be passed to this function.
%The first dimension is not required\footnote{A size can be specified here,
%but it is ignored by the compiler.}.
%
%Do not be confused by a function with this prototype:
%\begin{lstlisting}[stepnumber=0]{}
%  void f( int * a[ ] );
%\end{lstlisting}
%This defines a single dimensional array of integer pointers (which incidently
%can be used to create an array of arrays that acts much like a two dimensional
%array).
%
%For higher dimensional arrays, all but the first dimension of the array must
%be specified for parameters. For example, a four dimensional array parameter
%might be passed like:
%\begin{lstlisting}[stepnumber=0]{}
%  void f( int a[ ][4][3][2] );
%\end{lstlisting}
%\index{arrays!multidimensional!parameters|)}
\index{arrays!multidimensional|)}

\section{Array/String Instructions}
\index{string instructions|(} 

The 80x86 family of processors provide several instructions that are
designed to work with arrays. These instructions are called
\emph{string instructions}. They use the index registers (ESI and EDI)
to perform an operation and then to automatically increment or
decrement one or both of the index registers. The \emph{direction
flag} (DF) \index{register!FLAGS!DF} in the FLAGS register determines
where the index registers are incremented or decremented. There are
two instructions that modify the direction flag:
\begin{description}
\item[CLD] \index{CLD} clears the direction flag. In this state, the index registers
           are incremented.
\item[STD] \index{STD} sets the direction flag. In this state, the index registers are
           decremented.
\end{description}
A \emph{very} common mistake in 80x86 programming is to forget to explicitly
put the direction flag in the correct state. This often leads to code that
works most of the time (when the direction flag happens to be in the desired
state), but does not work \emph{all} the time.

%tested in string1.asm
\begin{figure}[t]
\centering
{\code
\begin{tabular}{|lp{1.5in}|lp{1.5in}|}
\hline
LODSB & AL = [ESI]\newline ESI = ESI $\pm$ 1 & 
STOSB & [EDI] = AL\newline EDI = EDI $\pm$ 1 \\
\hline
LODSW & AX = [ESI]\newline ESI = ESI $\pm$ 2 & 
STOSW & [EDI] = AX\newline EDI = EDI $\pm$ 2 \\
\hline
LODSD & EAX = [ESI]\newline ESI = ESI $\pm$ 4 & 
STOSD & [EDI] = EAX\newline EDI = EDI $\pm$ 4 \\
\hline
\end{tabular}
}
\caption{Reading and writing string instructions\label{fig:rwString}
         \index{LODSB} \index{STOSB} \index{LODSW} \index{LODSD} \index{STOSD}}
\end{figure}

%tested in string2.asm
\begin{figure}[t]
\begin{lstlisting}[language={[x86masm]Assembler},frame=single]
.data ; start definition of variables
string db "test",0
array1  dd  1, 2, 3, 4, 5, 6, 7, 8, 9, 10
array2  dd 10 dup (?)

.code ; start code portion of program
main proc ; start of the main procedure
    mov  eax,@data ; load the address of the data segment into eax
    mov  ds,eax ; load the address of the data segment into ds
    mov  es,eax ; load the address of the data segment into es
    ; the three previous instructions initalize the data segment and 

    ;copies array1 into array2
      cld      ; don't forget this!
      mov    esi, offset array1
      mov    edi, offset array2
      mov    ecx, lengthof array1
lp:
      lodsd  
      stosd
      loop  lp
\end{lstlisting}
\caption{Load and store example\label{fig:lodEx}}
\end{figure}

\subsection{Reading and writing memory}

The simplest string instructions either read or write memory or
both. They may read or write a byte, word or double word at a time.
Figure~\ref{fig:rwString} shows these instructions with a short
pseudo-code description of what they do. There are several points to
notice here. First, ESI is used for reading and EDI for writing. It is
easy to remember this if one remembers that SI stands for \emph{Source
Index} and DI stands for \emph{Destination
Index}. \index{register!ESI} \index{register!EDI} Next, notice that
the register that holds the data is fixed (either AL, AX or
EAX). Finally, note that the storing instructions use ES to determine
the segment to write to, not DS. In protected mode programming this is
not usually a problem, since there is only one data segment and ES
should be automatically initialized to reference it (just as DS
is). However, in real mode programming, it is \emph{very} important
for the programmer to initialize ES to the correct segment
\index{register!segment} selector value.  What this means to you is that 
you need to add {\code mov es,eax} to the beginning of your code (see 
line~10 in Figure~\ref{fig:lodEx}).  Figure~\ref{fig:lodEx} shows an example use of these
instructions that copies an array into another.

%\footnote{Another complication
%is that one can not copy the value of the DS register into the ES
%register directly using a single {\code MOV} instruction. Instead, the
%value of DS must be copied to a general purpose register (like AX) and
%then copied from that register to ES using two {\code MOV}
%instructions.}. 
%tested in string1.asm
\begin{figure}[t]
\centering
{\code
\begin{tabular}{|lp{2.5in}|}
\hline
MOVSB & byte [EDI] = byte [ESI] \newline ESI = ESI $\pm$ 1 \newline
        EDI = EDI $\pm$ 1 \\
\hline
MOVSW & word [EDI] = word [ESI] \newline ESI = ESI $\pm$ 2 \newline
        EDI = EDI $\pm$ 2 \\
\hline
MOVSD & dword [EDI] = dword [ESI] \newline ESI = ESI $\pm$ 4 \newline
        EDI = EDI $\pm$ 4 \\
\hline
\end{tabular}
}
\caption{Memory move string instructions\label{fig:movString} \index{MOVSB}
         \index{MOVSW} \index{MOVSD}}
\end{figure}

%tested in string1.asm
\begin{figure}[t]
\begin{lstlisting}[language={[x86masm]Assembler},frame=single]
          cld       ; don't forget this!
          mov    edi, offset array1
          mov    ecx, 10
          xor    eax, eax
          rep stosd
\end{lstlisting}
\caption{Zero array example\label{fig:zeroArrayEx}}
\end{figure}

The combination of a {\code LODSx} and {\code STOSx} instruction (as in
lines~19 and 20 of Figure~\ref{fig:lodEx}) is very common. In fact, this
combination can be performed by a single {\code MOVSx} string instruction.
Figure~\ref{fig:movString} describes the operations that these 
instructions perform. Lines~19 and 20 of Figure~\ref{fig:lodEx} could be
replaced with a single {\code MOVSD} instruction with the same effect. The
only difference would be that the EAX register would not be used at all
in the loop. %the previous statement was tested in string2.asm

\subsection{The {\code REP} instruction prefix\index{REP|(}}

The 80x86 family provides a special instruction prefix\footnote{A
instruction prefix is not an instruction, it is a special byte that is
placed before a string instruction that modifies the instructions
behavior. Other prefixes are also used to override segment defaults of
memory accesses} called {\code REP} that can be used with the above string
instructions. This prefix tells the CPU to repeat the next string instruction
a specified number of times. The ECX register is used to count the iterations
(just as for the {\code LOOP} instruction). Using the {\code REP} prefix, 
the loop in Figure~\ref{fig:lodEx} (lines~18 to 21) could be replaced with
a single line:
%tested in string2.asm
\begin{lstlisting}[language={[x86masm]Assembler},frame=none, numbers=none]
      rep movsd
\end{lstlisting}
Figure~\ref{fig:zeroArrayEx} shows another example that zeroes out the
contents of an array.
\index{REP|)}

%tested in string1.asm
\begin{figure}[t]
\centering
{\code
\begin{tabular}{|lp{3.5in}|}
\hline
CMPSB & compares byte [ESI] and byte [EDI] \newline ESI = ESI $\pm$ 1 
        \newline EDI = EDI $\pm$ 1 \\
\hline
CMPSW & compares word [ESI] and word [EDI] \newline ESI = ESI $\pm$ 2 
        \newline EDI = EDI $\pm$ 2 \\
\hline
CMPSD & compares dword [ESI] and dword [EDI] \newline ESI = ESI $\pm$ 4 
        \newline EDI = EDI $\pm$ 4 \\
\hline
SCASB & compares AL and [EDI] \newline EDI $\pm$ 1 \\
\hline
SCASW & compares AX and [EDI] \newline EDI $\pm$ 2 \\
\hline
SCASD & compares EAX and [EDI] \newline EDI $\pm$ 4 \\
\hline
\end{tabular}
}
\caption{Comparison string instructions\label{fig:cmpString}
         \index{CMPSB} \index{CMPSW} \index{CMPSD} \index{SCASB}
         \index{SCASW} \index{SCASD}}
\end{figure}

%tested in string3.asm part 1
\begin{figure}[t]
\begin{lstlisting}[language={[x86masm]Assembler},frame=single]
    cld
    mov    edi, offset array    ; pointer to start of array
    mov    ecx, lengthof array  ; number of elements
    mov    eax, 12       ; number to scan for
lp:
    scasd
    je     found
    loop   lp
    mov dx, offset notFoundStr
    jmp    onward
found:
    sub edi, 4  ; edi now points to 12 in array
    mov dx, offset foundStr
onward:
    mov ah, 9
    int 21h
\end{lstlisting}
\caption{Search example\label{fig:srchStrEx}}
\end{figure}

\subsection{Comparison string instructions}

Figure~\ref{fig:cmpString} shows several new string instructions that
can be used to compare memory with other memory or a register. They
are useful for comparing or searching arrays. They set the FLAGS
register just like the {\code CMP} instruction. The {\code CMPSx}
\index{CMPSB} \index{CMPSW} \index{CMPSD} instructions compare
corresponding memory locations and the {\code SCASx} \index{SCASB}
\index{SCASW} \index{SCASD} scan memory locations for a specific
value.

Figure~\ref{fig:srchStrEx} shows a short code snippet that searches
for the number 12 in a double word array. The {\code SCASD} instruction on
line~6 always adds 4 to EDI, even if the value
searched for is found. Thus, if one wishes to find the address of the 12
found in the array, it is necessary to subtract 4 from EDI (as 
line~12 does).

\begin{figure}[t]
\centering
\begin{tabular}{|l|p{4in}|}
\hline
{\code REPE}, {\code REPZ} & repeats instruction while Z flag is set or
                             at most ECX times \\
\hline
{\code REPNE}, {\code REPNZ} & repeats instruction while Z flag is cleared or
                             at most ECX times \\
\hline
\end{tabular}
\caption{{\code REPx} instruction prefixes\label{fig:repx} \index{REPE} 
          \index{REPNE}}
\end{figure}

%tested in string3.asm part 2
\begin{figure}
\begin{lstlisting}[language={[x86masm]Assembler},frame=single]
    cld
    mov    esi, offset block1 ; address of first block
    mov    edi, offset block2 ; address of second block
    mov    ecx, sizeof block1 ; size of blocks in bytes
    repe   cmpsb              ; repeat while Z flag is set
    je     equal              ; if Z set, blocks equal
    ; code to perform if blocks are not equal
    jmp    onward
equal:
   ; code to perform if equal
onward:
\end{lstlisting}
\caption{Comparing memory blocks\label{fig:cmpBlocksEx}}
\end{figure}

\subsection{The {\code REPx} instruction prefixes}

There are several other {\code REP}-like instruction prefixes that can be
used with the comparison string instructions. Figure~\ref{fig:repx} shows
the two new prefixes and describes their operation. {\code REPE} \index{REPE} and
{\code REPZ} are just synonyms for the same prefix (as are {\code REPNE} \index{REPNE}
and {\code REPNZ}). If the repeated comparison string instruction stops
because of the result of the comparison, the index register or registers
are still incremented and ECX decremented; however, the FLAGS register
still holds the state that terminated the repetition. 
\MarginNote{Why can one not just look to see if ECX is zero after the
repeated comparison?} Thus, it is possible
to use the Z flag to determine if the repeated comparisons stopped because
of a comparison or ECX becoming zero.

Figure~\ref{fig:cmpBlocksEx} shows an example code snippet that determines
if two blocks of memory are equal. The {\code JE} on 
line~6 of the example checks to see the result of the
previous instruction. If the repeated comparison stopped because it found
two unequal bytes, the Z flag will still be cleared and no branch is made;
however, if the comparisons stopped because ECX became zero, the Z flag
will still be set and the code branches to the {\code equal} label.

%\subsection{Example}


\index{string instructions|)}
\index{arrays|)}













