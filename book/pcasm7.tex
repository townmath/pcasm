% -*-latex-*-
\chapter{Structures}

\section{Structures\index{structures|(}}

\subsection{Introduction}

Structures are used in C to group together related data into a composite 
variable. This technique has several advantages:
\begin{enumerate}
\item It clarifies the code by showing that the data defined in the structure
      are intimately related.
\item It simplifies passing the data to functions. Instead of passing
      multiple variables separately, they can be passed as a single unit.
\item It increases the \index{locality}\emph{locality}\footnote{See the virtual memory 
management section of any Operating System text book for discussion of
this term.} of the code.
\end{enumerate}

From the assembly standpoint, a structure can be considered as an
array with elements of \emph{varying} size. The elements of real
arrays are always the same size and type. This property is what allows
one to calculate the address of any element by knowing the starting
address of the array, the size of the elements and the desired
element's index.

A structure's elements do not have to be the same size (and usually
are not). Because of this each element of a structure must be
explicitly specified and is given a \emph{tag} (or name) instead of a
numerical index.

In assembly, the element of a structure will be accessed in a similar
way as in C.  To access an element, you give the structure variable name
then a dot then the field.  

For example, consider the following structure shown in Figure~\ref{fig:CStructs}.
%tested in struct1.cpp
\begin{figure}[t]
\begin{lstlisting}[language=C,stepnumber=0]{}
struct exampleStruct {
  short int x=0;    /* 2-byte integer */
  int       y=1;    /* 4-byte integer */
  long int  z=2;    /* 8-byte integer   */
};

//down in main
exampleStruct s = {4,5,6};
\end{lstlisting}
\caption{Defining and declaring structs in C\label{fig:CStructs}}
\end{figure}

Compare this to Figure~\ref{fig:AsmStructs}, structures and 
structure variables are both declared in the data segment in assembly. 
%tested in struct1.asm
\begin{figure}[t]
\begin{lstlisting}[language={[x86masm]Assembler}]
.data
   exampleStruct struct
	   x dw 1
	   y dd 2
	   z dq 3
   exampleStruct ends
   s exampleStruct <4,5,6>
\end{lstlisting}
\caption{Defining and declaring structs in Assembly\label{fig:AsmStructs}}
\end{figure}
For both, if we wanted to access member y, we would use {\code s.y}.  If you only wanted
to specify values for x and z you can omit the y initializer and it will be assigned the default 
value: {\code s2 exampleStruct <7,,9>}.  

\subsection{Structure Example}
Here is an example of a structure that holds user data:
\lstinputlisting{../code/ch7/structEx.asm}
%maybe add more about structs, what else do they need to know? 

