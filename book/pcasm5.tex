%-*- latex -*-
\chapter{Subprograms}

This chapter looks at using subprograms to make modular programs and to
interface with high level languages (like C). Functions and procedures are
high level language examples of subprograms.

The code that calls a subprogram and the subprogram itself must agree
on how data will be passed between them. These rules on how data will
be passed are called \emph{calling conventions}. \index{calling
convention} A large part of this chapter will deal with the standard C
calling conventions that can be used to interface assembly subprograms
with C programs. This (and other conventions) often pass the addresses
of data (\emph{i.e.} pointers) to allow the subprogram to access the
data in memory.

\section{Indirect Addressing\index{indirect addressing|(}}

Indirect addressing allows registers to act like pointer variables. To
indicate that a register is to be used indirectly as a pointer, it is
enclosed in square brackets ({\code []}). For example:
%tested in subproc0.asm
\begin{lstlisting}[language={[x86masm]Assembler}]
      mov    ax, data     ; normal direct memory addressing of a word
      mov    ebx, offset data      ; ebx = & data
      mov    ax, [ebx]      ; ax = *ebx
\end{lstlisting}
\MarginNote{Note: Lines 1 and 3 perform the same operation.}
Because AX holds a word, line~3 reads a word starting at the address stored 
in EBX.  What EBX is assumed to point to is completely determined by what
instructions are used. Furthermore, even the fact that EBX is a pointer is
completely determined by the what instructions are used. If EBX is used
incorrectly, often there will be no assembler error; however, the program
will not work correctly. This is one of the many reasons that assembly
programming is more error prone than high level programming.

All the 32-bit general purpose (EAX, EBX, ECX, EDX) and index (ESI, EDI)
registers can be used for indirect addressing. 
\index{indirect addressing|)}

\section{Simple Subprogram Example\index{subprogram|(}}

A subprogram is an independent unit of code that can be used from different
parts of a program. In other words, a subprogram is like a function in C. A
jump can be used to invoke the subprogram, but returning presents a problem.
If the subprogram is to be used by different parts of the program, it must
return back to the section of code that invoked it. Thus, the jump back from
the subprogram can not be hard coded to a label. The code below shows how this
could be done using the indirect form of the {\code JMP} instruction. This 
form of the instruction uses the value of a register to determine where to
jump to (thus, the register acts much like a \emph{function pointer} in C.)
Here is a program that uses a subprogram to get a digit from the user.
\lstinputlisting{../code/ch5/subproc1.asm}

The {\code get\_digit} subprogram uses a simple, register-based calling
convention. It expects the EBX register to hold the address of the
BYTE to store the digit input into and the ECX register to hold the
code address of the instruction to jump back to. In line~22,
the {\code ret1} label is used to compute this return address. The {\code \$} 
operator could have been used to compute the return
address. The {\code \$} operator returns the current address for the
line it appears on. The expression {\code \$ + 8} computes the address
of the {\code ret1} label on line~24.

Both of these return address computations are awkward. The first method
requires a label to be defined for each subprogram call. The second method
does not require a label, but does require careful thought. If a near jump
was used instead of a short jump, the number to add to {\code \$} would not
be 8! Fortunately, there is a much simpler way to invoke subprograms. This
method uses the \emph{stack}.

\section{The Stack\index{stack|(}}

Many CPUs have built-in support for a stack. A stack is a Last-In First-Out
(\emph{LIFO}) list. The stack is an area of memory that is organized in this
fashion. The {\code PUSH} instruction adds data to the stack and the
{\code POP} instruction removes data. The data removed is always the last
data added (that is why it is called a last-in first-out list).

The SS segment register specifies the segment that contains the stack (usually
this is the same segment data is stored into). The ESP register contains the
address of the data that would be removed from the stack. This data is said
to be at the \emph{top} of the stack. Data can only be added in double word
units. That is, one can not push a single byte on the stack.

The {\code PUSH} instruction inserts a double word\footnote{Actually
words can be pushed too, but in 32-bit protected mode, it is easier to
work with only double words on the stack.} on the stack by subtracting
4 from ESP and then stores the double word at {\code [ESP]}. The
{\code POP} instruction reads the double word at {\code [ESP]} and
then adds 4 to ESP. The code below demonstrates how these instructions
work and assumes that ESP is initially {\code 0100H}.
%tested in stack.asm
\begin{lstlisting}[language={[x86masm]Assembler}]
    push   dword ptr 1   ; 1 stored at 00FCh, ESP = 00FCh
    push   dword ptr 2   ; 2 stored at 00F8h, ESP = 00F8h
    push   dword ptr 3   ; 3 stored at 00F4h, ESP = 00F4h
    pop    eax        ; EAX = 3, ESP = 00F8h
    pop    ebx        ; EBX = 2, ESP = 00FCh
    pop    ecx        ; ECX = 1, ESP = 0100h
\end{lstlisting}

The stack can be used as a convenient place to store data temporarily. It is
also used for making subprogram calls, passing parameters and local
variables.

The 80x86 also provides a {\code PUSHA} instruction (push all) that pushes the values
of EAX, EBX, ECX, EDX, ESI, EDI and EBP registers (not in this order). The
{\code POPA} instruction (pop all) can be used to pop them all back off.
\index{stack|)}

\section{The CALL and RET Instructions\index{subprogram!calling|(}}
\index{CALL|(}
\index{RET|(}
The 80x86 provides two instructions that use the stack to make calling
subprograms quick and easy. The CALL instruction makes an
unconditional jump to a subprogram and \emph{pushes} the address of
the next instruction on the stack. The RET instruction
\emph{pops off} an address and jumps to that address. When using these
instructions, it is very important that one manage the stack correctly
so that the right number is popped off by the RET instruction!

The previous program can be rewritten to use these new instructions by 
changing lines~21 to 24 to be:
%tested in subproc2.asm
\begin{lstlisting}[language={[x86masm]Assembler}, firstnumber=21]
      mov    ebx, offset input
      ; don't need this anymore
      call   get_digit
      ; this either
\end{lstlisting}
and change the subprogram {\code get\_digit} to:
\begin{lstlisting}[language={[x86masm]Assembler}]
get_digit:
    mov ah, 1
    int 21h
    and al, 0fh     ; char to int
    mov [ebx], al   ; store input into memory
    ret         ; jump back to caller
\end{lstlisting}

There are several advantages to CALL and RET:
\begin{itemize}
\item It is simpler!
\item It allows subprograms calls to be nested easily. Notice that
{\code get\_digit} could call {\code read\_char}. This call pushes another address
on the stack. At the end of {\code read\_char}'s code is a RET that pops
off the return address and jumps back to {\code get\_digit}'s code. Then when
{\code get\_digit}'s RET is executed, it pops off the return address that 
jumps back to {\code main}. This works correctly because of the LIFO
property of the stack.
%tested in subproc2.asm
\begin{lstlisting}[language={[x86masm]Assembler}]
get_digit:
    call read_char
    and al, 0fh     ; char to int
    mov [ebx], al   ; store input into memory
    ret         ; jump back to caller

read_char:
    mov ah, 1
    int 21h
    ret
\end{lstlisting}
\end{itemize}

Remember it is \emph{very} important to pop off all data that is pushed
on the stack. For example, consider the following:
%tested in subproc2.asm
\begin{lstlisting}[language={[x86masm]Assembler}]
get_digit:
    mov ah, 1
    int 21h
    and al, 0fh     ; char to int
    mov [ebx], al   ; store input into memory
    push eax
    ret         ; pops off eax value, not the return address!! 
\end{lstlisting}
This code would not return correctly!
\index{RET|)}
\index{CALL|)}

\section{Calling Conventions\index{calling convention|(}}

When a subprogram is invoked, the calling code and the subprogram (the
\emph{callee}) must agree on how to pass data between them. High-level
languages have standard ways to pass data known as \emph{calling 
conventions}. For high-level code to interface with assembly language, the
assembly language code must use the same conventions as the high-level
language. The calling conventions can differ from compiler to compiler or
may vary depending on how the code is compiled (\emph{e.g.} if
optimizations are on or not). One universal convention is that the code will
be invoked with a {\code CALL} instruction and return via a {\code RET}.

Calling conventions
allow one to create subprograms that are \emph{reentrant}. A reentrant
subprogram may be called at any point of a program safely (even inside
the subprogram itself).

\subsection{"Passing" parameters using registers}
As we saw above, the easiest way you can "pass" parameters is to use the registers.  
I say pass in quotes because you aren't doing anything extra like you do in C++.  Nevertheless,
you can use the existing values in your subprogram.  The changes
 you make will change the registers back in main, similar to passing by reference. 
 \subsubsection{Passing registers by value}
 Sometimes you might want to change the registers in the procedure, but not have those changes
 reflected in main.  This is similar to a local variable in C++ or passing a parameter by value.  There are two 
 common ways to do this, the first is to use push and pop to keep the registers from changing, 
 here is an example from earlier, but now the value of eax is preserved:
 %tested in subproc3.asm
 \begin{lstlisting}[language={[x86masm]Assembler}]
 get_digit:
    push eax
    mov ah, 1
    int 21h
    and al, 0fh     ; char to int
    mov [ebx], al   ; store input into memory
    pop eax
    ret        
\end{lstlisting}
The second way does the exact same thing, but without push and pop cluttering your code 
(the assembler adds those later).  To preserve the registers this way you use PROC with USES, 
this looks cleaner, and is explicit that EAX is being preserved.  
%tested in subproc3.asm
 \begin{lstlisting}[language={[x86masm]Assembler}]
get_digit proc uses eax
    mov ah, 1
    int 21h
    and al, 0fh     ; char to int
    mov [ebx], al   ; store input into memory
    ret         
get_digit endp
\end{lstlisting}
Note the ENDP at the bottom, this is needed to close the procedure.  We didn't need to do this when
we were using a named location for our procedure, but it becomes necessary when you use PROC.
Since we use this we also have to move {\code MAIN ENDP} above the procedure because we can't 
have nested procedures. 
\subsection{Passing parameters on the stack\index{stack|(}\index{stack!parameters|(}}

Parameters to a subprogram may be passed on the stack. They are pushed onto
the stack before the {\code CALL} instruction. Just as in C, if the
parameter is to be changed by the subprogram, the \emph{address} of the 
data must be passed, not the \emph{value}. If the parameter's size is less
than a double word, it must be converted to a double word before being pushed.

The parameters on the stack are not popped off by the subprogram, instead
they are accessed from the stack itself. Why?
\begin{itemize}
\item Since they have to be pushed on the stack before the {\code CALL}
instruction, the return address would have to be popped off first (and
then pushed back on again).
\item Often the parameters will have to be used in several places in the
subprogram. Usually, they can not be kept in a register for the entire
subprogram and would have to be stored in memory. Leaving them on the
stack keeps a copy of the data in memory that can be accessed at any
point of the subprogram.
\end{itemize}

\begin{figure}
\centering
\begin{tabular}{l|c|}
\cline{2-2}
&  \\ \cline{2-2}
ESP + 4 & Parameter \\ \cline{2-2}
ESP     & Return address \\ \cline{2-2}
 & \\ \cline{2-2}
\end{tabular}
\caption{}
\label{fig:stack1}
\end{figure}
Consider 
%\MarginNote{When using indirect addressing, the 80x86 processor 
%accesses different segments depending on what registers are used in the
%indirect addressing expression. ESP (and EBP) use the stack segment while
%EAX, EBX, ECX and EDX use the data seg\-ment. However, this is usually 
%unimportant for most protected mode programs, because for them the data 
%and stack segments are the same.}
a subprogram that is passed a single parameter on the stack. When
the subprogram is invoked, the stack looks like Figure~\ref{fig:stack1}.
The parameter can be accessed using indirect addressing ({\code [ESP+4]}
\footnote{It is legal to add a constant to a register when using indirect
addressing. More complicated expressions are possible too. This topic is covered
in the next chapter}).
\begin{figure}
\centering
\begin{tabular}{l|c|}
\cline{2-2}
&  \\ \cline{2-2}
ESP + 8 & Parameter \\ \cline{2-2}
ESP + 4 & Return address \\ \cline{2-2}
ESP     & subprogram data \\ \cline{2-2}
\end{tabular}
\caption{}
\label{fig:stack2}
\end{figure}

\begin{figure}[t]
 \begin{lstlisting}[language={[x86masm]Assembler}]
subprogram_label:
      push   ebp           ; save original EBP value on stack
      mov    ebp, esp      ; new EBP = ESP
; subprogram code
      pop    ebp           ; restore original EBP value
      ret
\end{lstlisting}
\caption{General subprogram form \label{fig:subskel1}}
\end{figure}

If the stack is also used inside the subprogram to store data, the
number needed to be added to ESP will change. For example,
Figure~\ref{fig:stack2} shows what the stack looks like if a DWORD is
pushed the stack. Now the parameter is at {\code ESP + 8} not {\code
ESP + 4}. Thus, it can be very error prone to use ESP when referencing
parameters. To solve this problem, the 80386 supplies another register
to use: EBP. This register's only purpose is to reference data on the
stack. The C calling convention mandates that a subprogram first save
the value of EBP on the stack and then set EBP to be equal to ESP.
This allows ESP to change as data is pushed or popped off the stack
without modifying EBP. At the end of the subprogram, the original
value of EBP must be restored (this is why it is saved at the start of
the subprogram.)  Figure~\ref{fig:subskel1} shows the general form of
a subprogram that follows these conventions.

\begin{figure}[t]
\centering
\begin{tabular}{ll|c|}
\cline{3-3}
&  & \\ \cline{3-3}
ESP + 8 & EBP + 8 & Parameter \\ \cline{3-3}
ESP + 4 & EBP + 4 & Return address \\ \cline{3-3}
ESP     & EBP     & saved EBP \\ \cline{3-3}
\end{tabular}
\caption{}
\label{fig:stack3}
\end{figure}


Lines 2 and 3 in Figure~\ref{fig:subskel1} make up the general \emph{prologue}
of a subprogram. Lines 5 and 6 make up the \emph{epilogue}. 
Figure~\ref{fig:stack3} shows what the stack looks like immediately
after the prologue. Now the parameter can be access with {\code [EBP + 8]}
at any place in the subprogram without worrying about what else has
been pushed onto the stack by the subprogram.

After the subprogram is over, the parameters that were pushed on the
stack must be removed. The C calling convention \index{calling
convention!C} specifies that the caller code must do this. Other
conventions are different. For example, the Pascal calling convention
\index{calling convention!Pascal} specifies that the subprogram must
remove the parameters.  (There is another form of the RET \index{RET}
instruction that makes this easy to do.) Some C compilers support this
convention too. 

\begin{figure}[t]
 \begin{lstlisting}[language={[x86masm]Assembler}]
      push   dword ptr 1        ; pass 1 as parameter
      call   funC
      add    esp, 4         ; remove parameter from stack
\end{lstlisting}
\caption{Sample C style subprogram call \label{fig:subcallC}}
\end{figure}

\begin{figure}[t]
 \begin{lstlisting}[language={[x86masm]Assembler}]
      push   dword ptr 1        ; pass 1 as parameter
      call   funPascal
      ...
      funPascal PROC
          ;subprogram code
          ret 4 ;remove parameter from stack
       funPascal ENDP
\end{lstlisting}
\caption{Sample Pascal style subprogram call \label{fig:subcallP}}
\end{figure}

Figure~\ref{fig:subcallC} shows how a subprogram using the C calling
convention would be called. Line~3 removes the parameter from the
stack by directly manipulating the stack pointer. A {\code POP}
instruction could be used to do this also, but would require the
useless result to be stored in a register. Actually, for this
particular case, many compilers would use a {\code POP ECX}
instruction to remove the parameter. The compiler would use a {\code
POP} instead of an {\code ADD} because the {\code ADD} requires more
bytes for the instruction. However, the {\code POP} also changes ECX's
value!  Figure~\ref{fig:subcallP} show how a subprogram using the Pascal
calling convention would remove the parameter from the stack. \\

Next is another example program with two subprograms that use
the C calling conventions discussed above. Line~64 (and other lines)
shows that multiple data and code segments may be declared in a single
source file. They will be combined into single data and code segments
in the linking process. Splitting up the data and code into separate
segments allow the data that a subprogram uses to be defined close by
the code of the subprogram.
\index{stack!parameters|)}

\lstinputlisting{../code/ch5/numSum.asm} %numSum


\subsection{Local variables on the stack\index{stack!local variables|(}}

The stack can be used as a convenient location for local variables. This is
exactly where C stores normal (or \emph{automatic} in C lingo) variables.
Using the stack for variables is important if one wishes subprograms to be
reentrant. A reentrant subprogram will work if it is invoked at any place,
including the subprogram itself. In other words, reentrant subprograms
can be invoked \emph{recursively}. Using the stack for variables also saves
memory. Data not stored on the stack is using memory from the beginning of
the program until the end of the program (C calls these types of variables
\emph{global} or \emph{static}). Data stored on the stack only use memory
when the subprogram they are defined for is active.

\begin{figure}[t]
 \begin{lstlisting}[language={[x86masm]Assembler}]
subprogram_label:
      push   ebp                ; save original EBP value on stack
      mov    ebp, esp           ; new EBP = ESP
      sub    esp, LOCAL_BYTES   ; = # bytes needed by locals
; subprogram code
      mov    esp, ebp           ; deallocate locals
      pop    ebp                ; restore original EBP value
      ret
\end{lstlisting}
\caption{General subprogram form with local variables\label{fig:subskel2}}
\end{figure}

\begin{figure}[t]
\begin{lstlisting}[language=C++,frame=tlrb]{}
void calc_sum( int n, int *sumP) {
  int i, sum = 0;

  for( i=1; i <= n; i++ ) {
    sum += i;
   }
  *sumP = sum;
}
\end{lstlisting}
\caption{C version of sum \label{fig:Csum}}
\end{figure}

%this is in subproc4.asm
\begin{figure}[t]
 \begin{lstlisting}[language={[x86masm]Assembler}]
; stack has the address of sumP and the value of n
cal_sum:
      push   ebp
      mov    ebp, esp
      sub    esp, 4               ; make room for local sum
      
      mov    dword ptr [ebp - 4], 0   ; sum = 0
      mov    ebx, 1               ; ebx (i) = 1
for_loop:
      cmp    ebx, [ebp+8]         ; is i <= n?
      jnle   end_for

      add    [ebp-4], ebx         ; sum += i
      inc    ebx
      jmp    short for_loop

end_for:
      mov    ebx, [ebp+12]        ; ebx = sumP
      mov    eax, [ebp-4]         ; eax = sum
      mov    [ebx], eax           ; *sumP = sum;

      mov    esp, ebp
      pop    ebp
      ret
 \end{lstlisting}
\caption{Assembly version of sum\label{fig:Asmsum}}
\end{figure}

Local variables are stored right after the saved EBP value in the stack.
They are allocated by subtracting the number of bytes required from ESP
in the prologue of the subprogram. Figure~\ref{fig:subskel2} shows the 
new subprogram skeleton. The EBP register is used to access local variables.
Consider the C function in Figure~\ref{fig:Csum}. Figure~\ref{fig:Asmsum}
shows how the equivalent subprogram could be written in assembly.

\begin{figure}[t]
\centering
\begin{tabular}{ll|c|}
\cline{3-3}
ESP + 16 & EBP + 12 & address of {\code sumP} \\ \cline{3-3}
ESP + 12 & EBP + 8  & {\code n} \\ \cline{3-3}
ESP + 8  & EBP + 4  & Return address \\ \cline{3-3}
ESP + 4  & EBP      & saved EBP \\ \cline{3-3}
ESP      & EBP - 4  & {\code sum} \\ \cline{3-3}
\end{tabular}
\caption{}
\label{fig:SumStack}
\end{figure}

Figure~\ref{fig:SumStack} shows what the stack looks like after the
prologue of the program in Figure~\ref{fig:Asmsum}. This section of
the stack that contains the parameters, return information and local
variable storage is called a \emph{stack frame}. Every invocation of
a C function creates a new stack frame on the stack.

\begin{figure}[t]
 \begin{lstlisting}[language={[x86masm]Assembler}]
subprogram_label:
      enter  LOCAL_BYTES, 0     ; = # bytes needed by locals
; subprogram code
      leave
      ret
 \end{lstlisting}
\caption{General subprogram form with local variables using 
{\code ENTER} and {\code LEAVE}\label{fig:subskel3}}
\end{figure}

%\MarginNote{Despite the fact that {\code ENTER} and {\code LEAVE} simplify
%the prologue and epilogue they are not used very often. Why? Because
%they are slower than the equivalent simpler instructions! This is an
%example of when one can not assume that a one instruction sequence is
%faster than a multiple instruction one.} 
The prologue and epilogue of a subprogram can be simplified by using
two special instructions that are designed specifically for this
purpose. The {\code ENTER} instruction performs the prologue code and the
{\code LEAVE} performs the epilogue. The {\code ENTER} instruction
takes two immediate operands. For the C calling convention, the second
operand is always 0. The first operand is the number of bytes needed by
local variables. The {\code LEAVE} instruction has no
operands. Figure~\ref{fig:factorial} shows how these instructions are
used. 

\index{stack!local variables|)}
\index{stack|)}
\index{calling convention|)}
\index{subprogram!calling|)}
\section{Multi-Module Programs\index{multi-module programs|(}}

A \emph{multi-module program} is one composed of more than one object
file.  The numSum.asm example program above is a multi-module
program. It consists of the Assembly driver object file and the assembly
object file. Recall that the linker
combines the object files into a single executable program. The linker
must match up references made to each label in one module (\emph{i.e.}
object file) to its definition in another module. In order for module
A to use a label defined in module B, the {\code extern} directive
must be used. After the {\code extern} \index{directive!extern}
directive comes a label and then {\code :proc} to let it know the label is a procedure. 
The directive tells the assembler to treat these labels as \emph{external} to the
module. That is, these are labels that can be used in this module, but
are defined in another. The {\code printInc.lib} file defines the
{\code printDec}, \emph{etc.} routines as external.

\index{multi-module programs|)}

\section{Reentrant and Recursive Subprograms\index{recursion|(}}

\index{subprogram!reentrant|(}
A reentrant subprogram must satisfy the following properties:
\begin{itemize}
\item It must not modify any code instructions. In a high level language
this would be difficult, but in assembly it is not hard for a program to
try to modify its own code. For example:
%this is tested in subproc5.asm
 \begin{lstlisting}[language={[x86masm]Assembler},numbers=none]
  mov word ptr [cs:$+9], 5    ; copy 5 into the word 7 bytes ahead
  add ax, 2               ; previous statement changes 2 to 5!
\end{lstlisting}
This code would work in real mode, but in protected mode operating systems 
the code segment is marked as read only. When the first line above executes,
the program will be aborted on these systems. This type of programming is
bad for many reasons. It is confusing, hard to maintain and does not allow
code sharing (see below).

\item It must not modify global data (such as data in the {\code data} and
the {\code bss} segments). All variables are stored on the stack.

\end{itemize}

There are several advantages to writing reentrant code.
\begin{itemize}
\item A reentrant subprogram can be called recursively.
\item A reentrant program can be shared by multiple processes. On many
multi-tasking operating systems, if there are multiple instances of a
program running, only \emph{one} copy of the code is in memory. Shared
libraries and DLL's (\emph{Dynamic Link Libraries}) use this idea as well.
\item Reentrant subprograms work much better in \emph{multi-threaded}
\footnote{A multi-threaded program has multiple threads of execution. That
is, the program itself is multi-tasked.} pro\-grams. Windows 9x/NT and most
UNIX-like operating systems (Solaris, Linux, \emph{etc.}) support 
multi-threaded programs.
\end{itemize}
\index{subprogram!reentrant|)}

\subsection{Recursive subprograms}

These types of subprograms call themselves. The recursion can be either
\emph{direct} or \emph{indirect}. Direct recursion occurs when a subprogram,
say {\code foo}, calls itself inside {\code foo}'s body. Indirect recursion
occurs when a subprogram is not called by itself directly, but by another
subprogram it calls. For example, subprogram {\code foo} could call
{\code bar} and {\code bar} could call {\code foo}.

Recursive subprograms must have a \emph{termination condition}. When
this condition is true, no more recursive calls are made. If a
recursive routine does not have a termination condition or the condition
never becomes true, the recursion will never end (much like an infinite
loop).

%also in subproc5.asm
\begin{figure}
 \begin{lstlisting}[language={[x86masm]Assembler}]
; finds n!
factorial proc
    enter 0,0 ; sets up stack frame
    mov    eax, [ebp+4] ; eax = n retrieve parameter from stack
    cmp    eax, 1 ; check termination condition
    jbe    term_cond    ; if n <= 1, terminate
    dec    eax    ; n--
    push   eax    ; call with (n-1)
    call   factorial ; eax = fact(n-1)
    mul    dword ptr [ebp+4]   ; edx:eax = eax * [ebp+4]
    term_cond:
    leave ; terminates stack frame
    ret 1 ; clean up the eax we pushed earlier
factorial endp
\end{lstlisting}
\caption{Recursive factorial function\label{fig:factorial}}
\end{figure}

\begin{figure}
\centering
%\includegraphics{factStack.eps}
\input{factStack.latex}
\caption{Stack frames for factorial function\label{fig:factStack}}
\end{figure}

Figure~\ref{fig:factorial} shows a function that calculates factorials
recursively. It could be called from C with:
\begin{lstlisting}[stepnumber=0]{}
x = factorial(3);         /* find 3! */
\end{lstlisting}
Figure~\ref{fig:factStack} shows what the stack looks like at its deepest
point for the above function call.


%Figures~\ref{fig:rec2C} and \ref{fig:rec2Asm} show another more
%complicated recursive example in C and assembly, respectively. What is
%the output is for {\code factorial(3)}? Note that the {\code ENTER} instruction
%creates a new {\code i} on the stack for each recursive call. Thus, each
%recursive instance of {\code f} has its own independent variable {\code i}.
%Defining {\code i} as a double word in the {\code data} segment would not
%work the same. 
\index{recursion|)}

\index{subprogram|)}
