% To create PDF version, type
%   pdflatex pcasm.tex
% This will produce errors the first time, type R at the error prompt
% Then rerun again (twice to get all the references.

\documentclass[11pt]{book}

%this seems to be more annoying than useful
%\typeout{-----------------------------------------}
%\typeout{Enter files to be included. (*=all)}
%\typeout{(pcasm1,pcasm2,...)}
%\typeout{-----------------------------------------}
%\typein[\infiles]{ }
%\if*\infiles\else\includeonly{\infiles}\fi


\newif\ifmypdf
\ifx\pdfoutput\undefined
    \pdffalse          % we are not running PDFLaTeX
\else
%    \pdfoutput=1       % we are running PDFLaTeX
%    \pdftrue
\fi
%----------------------------------------------------------------------
% The following is the construct that interests us in the end:
%\ifpdf
%   % Put PDF-specific stuff here
%\else
%   % Put LaTeX-specific stuff here
%\fi

\usepackage{indentfirst}  % indent first paragraph of sections
%\usepackage{graphicx}
\usepackage{listings}
\usepackage{epsfig}
\usepackage{longtable}
\usepackage{color}
\usepackage{makeidx}
\usepackage{amsmath}
\usepackage{float}

%\usepackage[pdftitle=PCASM,pdflang=en-US]{hyperref}%accessibility
%\usepackage[tagged]{accessibility}

\usepackage[pdftex,
            bookmarks=true,
            pdflang={en-US},
            pdftitle={PC Assembly Language},
            bookmarksnumbered=true,
            pdfauthor={Paul A. Carter Adapted for SCC's CISP310 by James R. Town},
            pdfsubject={80x86 Assembly Language Programming},
	    pdfkeywords={80x86 assembly programming}]{hyperref}
	 

\hypersetup{
   colorlinks,
    citecolor=black,
    filecolor=black,
    linkcolor=black,
    urlcolor=black
}



\definecolor{blue}{RGB}{0,0,255}
\definecolor{dkgreen}{RGB}{46,133,64}
\definecolor{gray}{rgb}{0.5,0.5,0.5}
\definecolor{mauve}{rgb}{0.58,0,0.82}

   
\author{Paul~A.~Carter \\Adapted for SCC's CISP310 by James R. Town}
\title{PC Assembly Language}
\usepackage{lecnote}
\makeindex


\hyphenation{num-bers SF OF CF DF fact dif-fer-ence inter-rupts op-er-ands}
\begin{document}
\maketitle
\newlength{\AsmMargin}
\setlength{\AsmMargin}{-1cm}
\DefineVerbatimEnvironment{AsmCodeListing}{Verbatim}
{numbers=left, frame=lines,xleftmargin=\AsmMargin, labelposition=all, commentchar=^ }

\newcommand{\MarginNote}[1]{\marginpar{\sloppy \em \small #1}}
\thispagestyle{empty}
\vspace*{\fill}
\noindent This work is licensed under the Creative Commons 
Attribution-NonCommercial-ShareAlike 4.0 International License. To view
a copy of this license, visit
{\code http://creativecommons.org/licenses/by-nc-sa/4.0/}.


\index{subroutine|see{subprogram}}
\index{REPNZ|see{REPNE}}
\index{REPZ|see{REPE}}
\index{C++!member functions|see{methods}}
\index{text segment|see{code segment}}

\vfill
\frontmatter
\include{toc}
%\include{pcasmf}
\mainmatter
% -*-latex-*-
\chapter{Introduction}
\section{Number Systems}

Memory in a computer consists of numbers. Computer memory does not
store these numbers in decimal (base 10). Because it greatly
simplifies the hardware, computers store all information in a binary
(base 2) format. First let's review the decimal system.

\subsection{Decimal\index{decimal}}

Base 10 numbers are composed of 10 possible digits (0-9). Each digit
of a number has a power of 10 associated with it based on its position
in the number. For example:
\begin{displaymath}
234 = 2 \times 10^2 + 3 \times 10^1 + 4 \times 10^0
\end{displaymath}

\subsection{Binary\index{binary|(}}

Base 2 numbers are composed of 2 possible digits (0 and 1). Each digit
of a number has a power of 2 associated with it based on its position
in the number. (A single binary digit is called a bit.) For
example\footnote{The 2 subscript is used to show that the number is
represented in binary, not decimal}:
\begin{eqnarray*}
11001_2 & = & 1 \times 2^4 + 1 \times 2^3 + 0 \times 2^2 + 0 \times 2^1
              + 1 \times 2^0 \\
 & = & 16 + 8 + 1 \\
 & = & 25
\end{eqnarray*}

This shows how binary may be converted to decimal. Table~\ref{tab:dec-bin}
shows how the first few numbers are represented in binary.
\begin{table}[t]
\begin{center}
\begin{tabular}{||c|c||cc||c|c||}
\hline
Decimal & Binary & & & Decimal & Binary \\
\hline
0       & 0000   & & & 8       & 1000 \\
\hline
1       & 0001   & & & 9       & 1001 \\
\hline
2       & 0010   & & & 10      & 1010 \\
\hline
3       & 0011   & & & 11      & 1011 \\
\hline
4       & 0100   & & & 12      & 1100 \\
\hline
5       & 0101   & & & 13      & 1101 \\
\hline
6       & 0110   & & & 14      & 1110 \\
\hline
7       & 0111   & & & 15      & 1111 \\
\hline
\end{tabular}
\caption{Decimal 0 to 15 in Binary\label{tab:dec-bin}}
\end{center}
\end{table}


\begin{figure}[h]
\begin{center}
%I think this example was confusing
%\begin{tabular}{|rrrrrrrrp{.1cm}|p{.1cm}rrrrrrrr|}
%\hline
%& \multicolumn{7}{c}{No previous carry} & & & \multicolumn{7}{c}{Previous carry} & \\
%\hline
%&  0 & &  0 & &  1 & &  1 & & &  0 & &  0 & &  1 & & 1  & \\
%& +0 & & +1 & & +0 & & +1 & & & +0 & & +1 & & +0 & & +1 &  \\
%\cline{2-2} \cline{4-4} \cline{6-6} \cline{8-8} \cline{11-11} \cline{13-13} \cline{15-15} \cline{17-17}
%& 0  & & 1  & & 1  & & 0  & & & 1  & & 0  & & 0  & & 1 & \\
%&    & &    & &    & & c  & & &    & & c  & & c  & & c & \\
%\hline
%\end{tabular}
\begin{tabular}{r}
 $11\overset{1}{0}\overset{1}{1}1_2$ \\
+$10001_2$ \\
\hline
$101100_2$ \\
\end{tabular}

\caption{Binary addition\label{fig:bin-add}}
\index{binary!addition}
\end{center}
\end{figure}

Figure~\ref{fig:bin-add} shows how individual binary digits ({\em
i.e.}, bits) are added. Here's an example:



If one considers the following decimal division:
\[ 1234 \div 10 = 123\; r\; 4 \]
he can see that this division strips off the rightmost decimal digit of
the number and shifts the other decimal digits one position to the right.
Dividing by two performs a similar operation, but for the binary digits
of the number. Consider the following binary division:
\[ 1101_2 \div 10_2 = 110_2\; r\; 1 \]
This fact can be used to convert a decimal
number to its equivalent binary representation as
Figure~\ref{fig:dec-convert} shows. This method finds the rightmost
digit first, this digit is called the \emph{least significant bit} (lsb).
The leftmost digit is called the \emph{most significant bit} (msb).
The basic unit of memory consists of 8 bits and is called a \emph{byte}.
\index{binary|)}

\begin{figure}[t]
%\centering
\fbox{\parbox{\textwidth}{
\begin{center}
\begin{tabular}{cc}
Decimal & Binary \\
$25 \div 2 = 12\;r\;1$ & $11001 \div 10 = 1100\;r\;1$ \\
$12 \div 2 = 6\;r\;0$  & $1100 \div 10 = 110\;r\;0$ \\
$6 \div 2 = 3\;r\;0$   & $110 \div 10 = 11\;r\;0$ \\
$3 \div 2 = 1\;r\;1$   & $11 \div 10 = 1\;r\;1$ \\
$1 \div 2 = 0\;r\;1$   & $1 \div 10 = 0\;r\;1$ \\
\end{tabular}
\end{center}
\centering
Thus $25_{10} = 11001_{2}$
}}
\caption{Decimal conversion to Binary \label{fig:dec-convert}}
\end{figure}

\subsection{Hexadecimal\index{hexadecimal|(}}

Hexadecimal numbers use base 16. Hexadecimal (or \emph{hex} for short) can be
used as a shorthand for binary numbers. Hex has 16 possible
digits. This creates a problem since there are no symbols to use for
these extra digits after 9. By convention, letters are used for these
extra digits. The 16 hex digits are 0-9 then A, B, C, D, E and F. The
digit A is equivalent to 10 in decimal, B is 11, etc. Each digit of a
hex number has a power of 16 associated with it. Example:
\begin{eqnarray*}
\rm
2BD_{16} & = & 2 \times 16^2 + 11 \times 16^1 + 13 \times 16^0 \\
         & = & 512 + 176 + 13 \\
         & = & 701 \\
\end{eqnarray*}
To convert from decimal to hex, use the same idea that was used for
binary conversion except divide by 16. See Figure~\ref{fig:hex-conv} for
an example.

\begin{figure}[t]
\fbox{\parbox{\textwidth}{
\begin{center}
\begin{tabular}{ccc}
$589 \div 16$ & = & $36\;r\;13$ \\
$36 \div 16$ & = & $2\;r\;4$ \\
$2 \div 16$ & = & $0\;r\;2$ \\
\end{tabular}
\end{center}
\centering
Thus $589 = 24\mathrm{D}_{16}$
}}
\caption{Decimal conversion to Hexadecimal \label{fig:hex-conv}}
\end{figure}

The reason that hex is useful is that there is a very simple way to
convert between hex and binary. Binary numbers get large and
cumbersome quickly. Hex provides a much more compact way to represent
binary.

To convert a hex number to binary, simply convert each hex digit to a
4-bit binary number. For example, $\mathrm{24D}_{16}$ is converted to
\mbox{$0010\;0100\; 1101_2$}. Note that the leading zeros of the
4-bits are important! If the leading zero for the middle digit of
$\mathrm{24D}_{16}$ is not used the result is wrong. Converting from
binary to hex is just as easy. One does the process in reverse. Convert
each 4-bit segments of the binary to hex. Start from the
right end, not the left end of the binary number. This ensures that
the process uses the correct 4-bit segments\footnote{If it is not
clear why the starting point makes a difference, try converting the
example starting at the left.}. Example:\newline

\begin{tabular}{cccccc}
$110$ & $0000$ & $0101$ & $1010$ & $0111$ & $1110_2$ \\
  $6$ & $0$    &   $5$  &   A  &  $7$   &    $\mathrm{E}_{16}$ \\
\end{tabular}\newline

A 4-bit number is called a \emph{nibble} \index{nibble}. Thus each hex
digit corresponds to a nibble. Two nibbles make a byte and so a byte
can be represented by a 2-digit hex number. A byte's value ranges from
0 to 11111111 in binary, 0 to FF in hex and 0 to 255 in decimal.
\index{hexadecimal|)}

\subsection{Octal\index{octal|(}}

Octal numbers use base 8. Less common than hexidecimal, octal can also be
used as a shorthand for binary numbers.  It is often used when the numbers
can be broken into groups of three bits.  The first fifteen numbers in octal are shown with their binary equivalents in Table~\ref{tab:oct-bin}.
\begin{table}[t]
\begin{center}
\begin{tabular}{||c|c||cc||c|c||}
\hline
Octal & Binary & & & Octal & Binary \\
\hline
0       & 000   & & & 10       & 001 000 \\
\hline
1       & 001   & & & 11       & 001 001 \\
\hline
2       & 010   & & & 12      & 001 010 \\
\hline
3       & 011   & & & 13      & 001 011 \\
\hline
4       & 100   & & & 14      & 001 100 \\
\hline
5       & 101   & & & 15      & 001 101 \\
\hline
6       & 110   & & & 16      & 001 110 \\
\hline
7       & 111   & & & 17      & 001 111 \\
\hline
\end{tabular}
\caption{Octal 0 to 17 in Binary\label{tab:oct-bin}}
\end{center}
\end{table}

\index{octal|)}

\section{Computer Organization}

\subsection{Memory\index{memory|(}}

The basic unit of memory is a byte. \index{byte} \MarginNote{Memory is
measured in units of kilobytes (~$2^{10} = 1,024$ bytes), megabytes
(~$2^{20} = 1,048,576$ bytes) and gigabytes (~$2^{30} = 1,073,741,824$
bytes).}A computer with 32 megabytes of memory can hold roughly 32
million bytes of information. Each byte in memory is labeled by a
unique number known as its address as Figure~\ref{fig:memory} shows.

\begin{figure}[ht]
\begin{center}
\begin{tabular}{rcccccccc}
Address & 0 & 1 & 2 & 3 & 4 & 5 & 6 & 7 \\
\cline{2-9}
Memory & \multicolumn{1}{|c}{2A}  & \multicolumn{1}{|c}{45}
       & \multicolumn{1}{|c}{B8} & \multicolumn{1}{|c}{20}
       & \multicolumn{1}{|c}{8F} & \multicolumn{1}{|c}{CD}
       & \multicolumn{1}{|c}{12} & \multicolumn{1}{|c|}{2E} \\
\cline{2-9}
\end{tabular}
\caption{ Memory Addresses \label{fig:memory} }
\end{center}
\end{figure}

\begin{table}[t]
\begin{center}
\begin{tabular}{|l|l|}
\hline
Term & Size \\ \hline
nibble & 1/2 bytes \\ \hline
byte & 1 byte \\ \hline
word & 2 bytes \\ \hline
double word & 4 bytes \\ \hline
quad word & 8 bytes \\ \hline
paragraph & 16 bytes \\ \hline
\end{tabular}
\caption{ Units of Memory \label{tab:mem_units} }
\end{center}
\end{table}

Often memory is used in larger chunks than single bytes. On
the PC architecture, names have been given to these larger sections of
memory as Table~\ref{tab:mem_units} shows.

All data in memory is numeric. Characters are stored by using a
\emph{character code} that maps numbers to characters. One of the
most common character codes is known as \emph{ASCII} (American
Standard Code for Information Interchange). A new, more complete code
that is supplanting ASCII is Unicode. One key difference between the
two codes is that ASCII uses one byte to encode a character, but
Unicode uses multiple bytes. There are several different forms of Unicode.
On x86 C/C++ compilers, Unicode is represented in code using the
{\code wchar\_t} type and the UTF-16 encoding which uses 16 bits (or a
\emph{word}) per character. For example, ASCII maps the byte $41_{16}$
($65_{10}$) to the character capital \emph{A}; UTF-16 maps it to the
word $0041_{16}$. Since ASCII uses a byte, it is limited to only 256
different characters\footnote{In fact, ASCII only uses the lower 7-bits
and so only has 128 different values to use.}. Unicode extends the ASCII
values and allows many more characters to be represented. This is important
for representing characters for all the languages of the world.
\index{memory|)}

\subsection{The CPU\index{CPU|(}}

The Central Processing Unit (CPU) is the physical device that performs
instructions. The instructions that CPUs perform are generally very
simple. Instructions may require the data they act on to be in special
storage locations in the CPU itself called
\emph{registers}. \index{register} The CPU can access data in registers
much faster than data in memory. However, the number of registers in a
CPU is limited, so the programmer must take care to keep only
currently used data in registers.

The instructions a type of CPU executes make up the CPU's
\emph{machine language}. \index{machine language} Machine programs
have a much more basic structure than higher-level languages. Machine
language instructions are encoded as raw numbers, not in friendly text
formats. A CPU must be able to decode an instruction's purpose very
quickly to run efficiently. Machine language is designed with this
goal in mind, not to be easily deciphered by humans. Programs written
in other languages must be converted to the native machine language of
the CPU to run on the computer. A \emph{compiler} \index{compiler} is
a program that translates programs written in a programming language
into the machine language of a particular computer architecture. In
general, every type of CPU has its own unique machine language. This
is one reason why programs written for a Mac can not run on an
IBM-type PC.

Computers use a \emph{clock} \index{clock} to synchronize the
execution of the \MarginNote{\emph{GHz} stands for gigahertz or one
billion cycles per second.  A 1.5 GHz CPU has 1.5 billion clock pulses
per second.} instructions.  The clock pulses at a fixed frequency
(known as the \emph{clock speed}). When you buy a 1.5 GHz computer,
1.5 GHz is the frequency of this clock\footnote{Actually, clock pulses
are used in many different components of a computer. The other
components often use different clock speeds than the CPU.}. The clock
does not keep track of minutes and seconds. It simply beats at a
constant rate. The electronics of the CPU uses the beats to perform
their operations correctly, like how the beats of a metronome help one
play music at the correct rhythm.  The number of beats (or as they are
usually called \emph{cycles}) an instruction requires depends on the
CPU generation and model. The number of cycles depends on the
instructions before it and other factors as well.


\subsection{The 80x86 family of CPUs\index{CPU!80x86}}

IBM-type PC's contain a CPU from Intel's 80x86 family (or a clone of one).
The CPU's in this family all have some common features including a base
machine language. However, the more recent members greatly enhance the
features.
\begin{description}

\item[8088,8086:] These CPU's from the programming standpoint are
identical. They were the CPU's used in the earliest PC's. They provide
several 16-bit registers: AX, BX, CX, DX, SI, DI, BP, SP, CS, DS, SS,
ES, IP, FLAGS. They only support up to one megabyte of memory and only
operate in \emph{real mode}.  In this mode, a program may access any
memory address, even the memory of other programs! This makes
debugging and security very difficult! Also, program memory has to be
divided into \emph{segments}. Each segment can not be larger than
64K.

\item[80286:] This CPU was used in AT class PC's. It adds some new
instructions to the base machine language of the 8088/86.  However,
its main new feature is \emph{16-bit protected mode}.  In this mode,
it can access up to 16 megabytes and protect programs from accessing
each other's memory. However, programs are still divided into
segments that could not be bigger than 64K.

\item[80386:] This CPU greatly enhanced the 80286. First, it extends many
of the registers to hold 32-bits (EAX, EBX, ECX, EDX, ESI, EDI, EBP, ESP,
EIP) and adds two new 16-bit registers FS and GS. It also adds a new
\emph{32-bit protected mode}. In this mode, it can access up to 4 gigabytes.
Programs are again divided into segments, but now each segment can also be
up to 4 gigabytes in size!

\item[80486/Pentium/Pentium Pro:] These members of the 80x86 family add
very few new features. They mainly speed up the execution of the
instructions.

\item[Pentium MMX:] This processor adds the MMX (MultiMedia eXtensions)
instructions to the Pentium. These instructions can speed up common graphics
operations.

\item[Pentium II:] This is the Pentium Pro processor with the MMX instructions
added. (The Pentium III is essentially just a faster Pentium II.)

\end{description}
\index{CPU|)}

\subsection{8086 16-bit Registers\index{register|(}}

The original 8086 CPU provided four 16-bit general purpose registers:
AX, BX, CX and DX. Each of these registers could be decomposed into
two 8-bit registers. For example, the AX register could be decomposed
into the AH and AL registers as Figure~\ref{fig:AX_reg} shows. The AH
register contains the upper (or high) 8 bits of AX and AL contains the
lower 8 bits of AX. Often AH and AL are used as independent one byte
registers; however, it is important to realize that they are not
independent of AX. Changing AX's value will change AH and AL and
{\em vice versa}\/. The general purpose registers are used in many of
the data movement and arithmetic instructions.

\begin{figure}
\begin{center}
\begin{tabular}{cc}
\multicolumn{2}{c}{AX} \\
\hline
\multicolumn{1}{||c|}{AH} & \multicolumn{1}{c||}{AL} \\
\hline
\end{tabular}
\caption{The AX register \label{fig:AX_reg} }
\end{center}
\end{figure}

There are two 16-bit index registers\index{register!index}: SI and
DI. They are often used as pointers, but can be used for many of the
same purposes as the general registers. However, they can not be
decomposed into 8-bit registers.

The 16-bit BP and SP registers are used to point to data in the
machine language stack and are called the Base Pointer\index{register!base pointer}
and Stack Pointer\index{register!stack pointer}, respectively. These will be discussed later.

The 16-bit CS, DS, SS and ES registers are \emph{segment
registers}. \index{register!segment} They denote what memory is used
for different parts of a program. CS stands for Code Segment, DS for
Data Segment, SS for Stack Segment and ES for Extra Segment. ES is
used as a temporary segment register. The details of these registers
are in Sections~\ref{real_mode} and \ref{16prot_mode}.

The Instruction Pointer (IP) \index{register!IP} register is used with
the CS register to keep track of the address of the next instruction
to be executed by the CPU. Normally, as an instruction is executed, IP
is advanced to point to the next instruction in memory.

The FLAGS \index{register!FLAGS} register stores important information
about the results of a previous instruction. These results are stored
as individual bits in the register. For example, the Z bit is 1 if the
result of the previous instruction was zero or 0 if not zero. Not all
instructions modify the bits in FLAGS, consult the table in the
appendix to see how individual instructions affect the FLAGS register.

\subsection{80386 32-bit registers\index{register!32-bit}}

The 80386 and later processors have extended registers. For example, the
16-bit AX register is extended to be 32-bits. To be backward compatible, AX
still refers to the 16-bit register and EAX is used to refer to the extended
32-bit register. AX is the lower 16-bits of EAX just as AL is the lower 8-bits
of AX (and EAX). There is no way to access the upper 16-bits of EAX directly.
The other extended registers are EBX, ECX, EDX, ESI and EDI.

Many of the other registers are extended as well. BP becomes
EBP\index{register!base pointer}; SP becomes ESP\index{register!stack
pointer}; FLAGS becomes EFLAGS\index{register!EFLAGS} and IP becomes
EIP\index{register!EIP}. However, unlike the index and general purpose
registers, in 32-bit protected mode (discussed below) only the
extended versions of these registers are used.

The segment registers are still 16-bit in the 80386. There are also
two new segment registers: FS and GS\index{register!segment}. Their
names do not stand for anything. They are extra temporary segment
registers (like ES).

One of definitions of the term \emph{word} \index{word} refers to the
size of the data registers of the CPU. For the 80x86 family, the term
is now a little confusing. In Table~\ref{tab:mem_units}, one sees that
\emph{word} is defined to be 2 bytes (or 16 bits). It was given this
meaning when the 8086 was first released. When the 80386 was
developed, it was decided to leave the definition of \emph{word}
unchanged, even though the register size changed.
\index{register|)}

\subsection{Real Mode \label{real_mode} \index{real mode|(}}

In \MarginNote{So where did the infamous DOS 640K limit come from? The BIOS
required some of the 1M for its code and for hardware devices like the video
screen.} real mode, memory is limited to only one megabyte ($2^{20}$ bytes).
Valid
address range from (in hex) 00000 to FFFFF.\@  % \@ means end of sentence
These addresses require a 20-bit number. Obviously, a 20-bit number will not
fit into any of the 8086's 16-bit registers. Intel solved this problem, by
using two 16-bit values to determine an address. The first 16-bit value is called
the \emph{selector}. Selector values must be stored in segment registers. The
second 16-bit value is called the \emph{offset}. The physical address
referenced by a 32-bit \emph{selector:offset} pair is computed by the formula
\[ 16 * {\rm selector} + {\rm offset} \]
Multiplying by 16 in hex is easy, just add a 0 to the right of the number. For
example, the physical addresses referenced by 047C:0048 is given by:
\begin{center}
\begin{tabular}{r}
047C0 \\
+0048 \\
\hline
04808 \\
\end{tabular}
\end{center}
In effect, the selector value is a paragraph number
(see Table~\ref{tab:mem_units}).

Real segmented addresses have disadvantages:
\begin{itemize}
\item A single selector value can only reference 64K of memory (the
upper limit of the 16-bit offset). What if a program has more than 64K
of code? A single value in CS can not be used for the entire execution
of the program.  The program must be split up into sections (called
\emph{segments}\index{memory!segments}) less than 64K in size. When
execution moves from one segment to another, the value of CS must be
changed. Similar problems occur with large amounts of data and the DS
register. This can be very awkward!

\item Each byte in memory does not have a unique segmented address. The
physical address 04808 can be referenced by 047C:0048, 047D:0038, 047E:0028
or 047B:0058.\@ This can complicate the comparison of segmented addresses.

\end{itemize}
\index{real mode|)}

\subsection{16-bit Protected Mode \label{16prot_mode} \index{protected mode!16-bit|(}}

In the 80286's 16-bit protected mode, selector values are interpreted
completely differently than in real mode. In real mode, a selector
value is a paragraph number of physical memory. In protected mode, a
selector value is an \emph{index} into a \emph{descriptor table}. In
both modes, programs are divided into
segments\index{memory:segments}. In real mode, these segments are at
fixed positions in physical memory and the selector value denotes the
paragraph number of the beginning of the segment. In protected mode,
the segments are not at fixed positions in physical memory. In fact,
they do not have to be in memory at all!

Protected mode uses a technique called \emph{virtual memory}
\index{memory!virtual}. The basic idea of a virtual memory system is
to only keep the data and code in memory that programs are currently
using. Other data and code are stored temporarily on disk until they
are needed again.  In 16-bit protected mode, segments are moved
between memory and disk as needed. When a segment is returned to
memory from disk, it is very likely that it will be put into a
different area of memory that it was in before being moved to
disk. All of this is done transparently by the operating system. The
program does not have to be written differently for virtual memory to
work.

In protected mode, each segment is assigned an entry in a descriptor table.
This entry has all the information that the system needs to know about the
segment. This information includes: is it currently in memory; if in memory,
where is it; access permissions ({\em e.g.\/}, read-only). The index of the
entry of the segment is the selector value that is stored in segment registers.

One \MarginNote{One well-known PC columnist called the 286 CPU ``brain
dead.''} big disadvantage of 16-bit protected mode is that offsets
are still 16-bit quantities. As a consequence of this, segment sizes
are still limited to at most 64K. This makes the use of large arrays
problematic!
\index{protected mode!16-bit|)}

\subsection{32-bit Protected Mode\index{protected mode!32-bit|(}}

The 80386 introduced 32-bit protected mode. There are two major differences
between 386 32-bit and 286 16-bit protected modes:
\begin{enumerate}
\item

Offsets are expanded to be 32-bits. This allows an offset to range up
to 4 billion. Thus, segments can have sizes up to 4 gigabytes.

\item

Segments\index{memory!segments} can be divided into smaller 4K-sized
units called \emph{pages}\index{memory!pages}. The virtual
memory\index{memory!virtual} system works with pages now instead of
segments. This means that only parts of segment may be in memory at
any one time. In 286 16-bit mode, either the entire segment is in
memory or none of it is. This is not practical with the larger
segments that 32-bit mode allows.

\end{enumerate}

\index{protected mode!32-bit|)}

In Windows 3.x, \emph{standard mode} referred to 286 16-bit protected mode and
\emph{enhanced mode} referred to 32-bit mode. Windows 9X, Windows NT/2000/XP, OS/2
and Linux all run in paged 32-bit protected mode.

\subsection{Interrupts\index{interrupt}}

Sometimes the ordinary flow of a program must be interrupted to process events
that require prompt response. The hardware of a computer provides a mechanism
called \emph{interrupts} to handle these events. For example, when a mouse is
moved, the mouse hardware interrupts the current program to handle the mouse
movement (to move the mouse cursor, {\em etc.\/}) Interrupts cause control to
be passed to an \emph{interrupt handler}. Interrupt handlers are routines that
process the interrupt. Each type of interrupt is assigned an integer number.
At the beginning of physical memory, a table of \emph{interrupt vectors}
resides that contain the segmented addresses of the interrupt handlers. The
number of interrupt is essentially an index into this table.

External interrupts are raised from outside the CPU. (The mouse is an
example of this type.) Many I/O devices raise interrupts ({\em
e.g.\/}, keyboard, timer, disk drives, CD-ROM and sound
cards). Internal interrupts are raised from within the CPU, either
from an error or the interrupt instruction. Error interrupts are also
called \emph{traps}. Interrupts generated from the interrupt
instruction are called \emph{software interrupts}. DOS uses these types of
interrupts to implement its API (Application Programming Interface). More
modern operating systems (such as Windows and UNIX) use a C based interface.
\footnote{However, they may use a lower level interface at the kernel level.}

Many interrupt handlers return control back to the interrupted program
when they finish. They restore all the registers to the same values
they had before the interrupt occurred. Thus, the interrupted program
runs as if nothing happened (except that it lost some CPU
cycles). Traps generally do not return. Often they abort the program.

\section{Assembly Language}

\subsection{Machine language\index{machine language}}

Every type of CPU understands its own machine language. Instructions
in machine language are numbers stored as bytes in memory. Each
instruction has its own unique numeric code called its \emph{operation
code} or \emph{opcode} \index{opcode} for short. The 80x86 processor's
instructions vary in size.  The opcode is always at the beginning of
the instruction. Many instructions also include data ({\em e.g.\/},
constants or addresses) used by the instruction.

Machine language is very difficult to program in directly. Deciphering the
meanings of the numerical-coded instructions is tedious for humans. For
example, the instruction that says to add the EAX and EBX registers together
and store the result back into EAX is encoded by the following hex codes:
\begin{quote}
   03 C3
\end{quote}
This is hardly obvious. Fortunately, a program called an
\emph{assembler} \index{assembler} can do this tedious work for the
programmer.

\subsection{Assembly language\index{assembly language|(}}

An assembly language program is stored as text (just as a higher level language
program). Each assembly instruction represents exactly one machine instruction.
For example, the addition instruction described above would be represented
in assembly language as:
\begin{CodeQuote}
   add eax, ebx
\end{CodeQuote}
Here the meaning of the instruction is \emph{much} clearer than in
machine code. The word {\code add} is a \emph{mnemonic}
\index{mnemonic} for the addition instruction.  The general form of an
assembly instruction is:
\begin{CodeQuote}
  {\em mnemonic operand(s)}
\end{CodeQuote}

An \emph{assembler} \index{assembler} is a program that reads a text
file with assembly instructions and converts the assembly into machine
code.  \emph{Compilers} \index{compiler} are programs that do similar
conversions for high-level programming languages. An assembler is much
simpler than a compiler. \MarginNote{It took several years for
computer scientists to figure out how to even write a compiler!} Every
assembly language statement directly represents a single machine
instruction. High-level language statements are \emph{much} more
complex and may require many machine instructions.

Another important difference between assembly and high-level languages is that
since every different type of CPU has its own machine language, it also has
its own assembly language. Porting assembly programs between different computer
architectures is \emph{much} more difficult than in a high-level language.

This book's examples uses the Microsoft's Assembler or MASM \index{MASM}
for short. It is freely available off the Internet. More common assemblers are Netwide Assembler (NASM)
\index{NASM} or Borland's Assembler (TASM). \index{TASM} There are
some differences in the assembly syntax for MASM/TASM and NASM, but the
big ideas are the same.

\subsection{Instruction operands}

Machine code instructions have varying number and type of operands; however,
in general, each instruction itself will have a fixed number of operands (0
to 3). Operands can have the following types:
\begin{description}
\item[register:]
These operands refer directly to the contents of the CPU's registers.
\item[memory:]
These refer to data in memory. The address of the data may be a constant
hardcoded into the instruction or may be computed using values of registers.
Address are always offsets from the beginning of a segment.
\item[immediate:]
\index{immediate}
These are fixed values that are listed in the instruction itself. They are
stored in the instruction itself (in the code segment), not in the data
segment.
\item[implied:]
These operands are not explicitly shown. For example, the increment
instruction adds one to a register or memory. The one is implied.
\end{description}
\index{assembly language|)}

\subsection{Basic instructions}

The most basic instruction is the {\code MOV} \index{MOV} instruction. It moves
data from one location to another (like the assignment operator in a
high-level language). It takes two operands:
\begin{CodeQuote}
  mov {\em dest, src}
\end{CodeQuote}
The data specified by {\em src} is copied to {\em dest\/}. One restriction
is that both operands may not be memory operands. This points out another
quirk of assembly. There are often somewhat arbitrary rules about how the
various instructions are used. The operands must also be the same size. The
value of AX can not be stored into BL.

\lstset{%frame=tb,
   language=[x86masm]Assembler,
   morekeywords={.code},
   aboveskip=3mm,
   belowskip=3mm,
   showstringspaces=false,
   columns=flexible,
   basicstyle={\small\ttfamily},
   numbers=left,%none,%
   numberstyle=\tiny\color{gray},
   keywordstyle=\color{blue},
   commentstyle=\color{dkgreen},
   stringstyle=\color{mauve},
   breaklines=true,
   frame=none,
   breakatwhitespace=true,
   tabsize=3
   }

Here is an example (semicolons start a comment\index{comment}):
%\begin{AsmCodeListing}[frame=none, numbers=none]
\begin{lstlisting}[language={[x86masm]Assembler}]
  mov    eax, 3   ; store 3 into EAX register (3 is immediate operand)
  mov    bx, ax   ; store the value of AX into the BX register
\end{lstlisting}
%AsmCodeListing}

The {\code ADD} \index{ADD} instruction is used to add integers.
\begin{lstlisting}[language={[x86masm]Assembler}]
  add    eax, 4   ; eax = eax + 4
  add    al, ah   ; al = al + ah
\end{lstlisting}

The {\code SUB} \index{SUB} instruction subtracts integers.
\begin{lstlisting}[language={[x86masm]Assembler}]
  sub    bx, 10   ; bx = bx - 10
  sub    ebx, edi ; ebx = ebx - edi
\end{lstlisting}

The {\code INC} \index{INC} and {\code DEC} \index{DEC} instructions
increment or decrement values by one. Since the one is an implicit
operand, the machine code for {\code INC} and {\code DEC} is smaller
than for the equivalent {\code ADD} and {\code SUB} instructions.
\begin{lstlisting}[language={[x86masm]Assembler}]
  inc    ecx      ; ecx++
  dec    dl       ; dl--
\end{lstlisting}

\subsection{Directives\index{directive|(}}

A \emph{directive} is an artifact of the assembler not the CPU. They are
generally used to either instruct the assembler to do something or inform
the assembler of something. They are not translated into machine code. Common
uses of directives are:
\begin{list}{$\bullet$}{\setlength{\itemsep}{0pt}}
\item define constants
\item define memory to store data into
\item group memory into segments
\item conditionally include source code
\item include other files
\end{list}

\subsubsection{The equ directive\index{directive!equ}}

The {\code equ} directive can be used to define a \emph{symbol}. Symbols are
named constants that can be used in the assembly program. The format is:
\begin{quote}
  \code {\em symbol} equ {\em value}
\end{quote}
Symbol values can \emph{not} be redefined later.

\subsubsection{Data directives\index{directive!data|(}}

\begin{table}[t]
\centering
\begin{tabular}{||c|c||} \hline
{\bf Unit} & {\bf Letter} \\
\hline
byte & B \\
word & W \\
double word & D \\
quad word & Q \\
ten bytes & T \\
\hline
\end{tabular}
\caption{Letters for {\code DX} Directives
         \label{tab:size-letters} }
\end{table}

Data directives are used in data segments to define room for
memory. There are two ways memory can be reserved. The first way only
defines room for data; the second way defines room and an initial
value. You can use a ? to reserve uninitialized memory for a given size,
or you can give the memory location an initial value.
%The first method uses one of the {\code RES{\em
%X}}\index{directive!RES\emph{X}} directives. The {\em X} is replaced
%with a letter that determines the size of the object (or objects) that
%will be stored.
Table~\ref{tab:size-letters} shows the possible
values for the different sizes of data.

%The second method (that defines an initial value, too) uses one of the
%{\code D{\em X}} directives\index{directive!D\emph{X}}. The {\em X}
%letters are the same as those in the {\code RES{\em X}} directives.


It is very common to mark memory locations with
\emph{labels}. \index{label} Labels allow one to easily refer to
memory locations in code. Below are several examples:
\begin{lstlisting}[language={[x86masm]Assembler}]
L1  db  0        ; byte labeled L1 with initial value 0
L2  dw  1000     ; word labeled L2 with initial value 1000
L3  db  110101b  ; byte initialized to binary 110101 (53 in decimal)
L4  db  12h      ; byte initialized to hex 12 (18 in decimal)
L5  db  17o      ; byte initialized to octal 17 (15 in decimal)
L6  dd  1A92h    ; double word initialized to hex 1A92
L7  db  ?        ; 1 uninitialized byte
L8  db  "A"      ; byte initialized to ASCII code for A (65)
\end{lstlisting}

Double quotes and single quotes are treated the same. Consecutive data
definitions are stored sequentially in memory. That is, the word L2 is
stored immediately after L1 in memory. Sequences of memory may also be
defined, which we'll see later can be treated as arrays.
\begin{lstlisting}[language={[x86masm]Assembler}]
L9   db    0, 1, 2, 3              ; defines 4 bytes
L10  db    "w", "o", "r", 'd', 0   ; defines a C string = "word"
L11  db    'word', 0                ; same as L10
\end{lstlisting}

The {\code DD}\index{directive!DD} directive can be used to define
both integer and single precision floating point\footnote{Single
precision floating point is equivalent to a {\code float} variable in
C.} constants. However, the {\code DQ}\index{directive!DQ} can only
be used to define double precision floating point constants.

For large sequences,  {\code DUP} \index{directive!DUP}
directive is often useful. This directive repeats its operand a
specified number of times. For example,
\begin{lstlisting}[language={[x86masm]Assembler}]
L12  db 100 dup (0)                 ; equivalent to 100 (db 0)'s
L13  dw 100 dup (?)                 ; reserves room for 100 words
\end{lstlisting}
\index{directive!data|)}
\index{directive|)}

\index{label|(}
Remember that labels can be used to refer to data in code. There are
two ways that a label can be used. If a plain label is used, it is
interpreted as the data at the address (or offset) of the label. If the label is
used with the offset operator, it is interpreted as the address of the data.  You might notice that labels act like variables in higher level languages.
In 32-bit mode, addresses are 32-bit. Here are some examples:
\begin{lstlisting}[language={[x86masm]Assembler}]
mov  al, L1      ; copy byte at L1 into AL
mov  eax, offset L1    ; EAX = address of byte at L1
mov  L1, ah      ; copy AH into byte at L1
mov  eax, L6     ; copy double word at L6 into EAX
add  eax, L6     ; EAX = EAX + double word at L6
add  L6, eax     ; double word at L6 += EAX
mov  al, byte ptr L6   ; copy first byte of double word at L6 into AL
\end{lstlisting}
Line 7 of the examples shows an important property of assembly. The assembler does
\emph{not} keep track of the type of data that a label refers to. It is up
to the programmer to make sure that he (or she) uses a label correctly. Later
it will be common to store addresses of data in registers and use the register
like a pointer variable in C. Again, no checking is made that a pointer is
used correctly. In this way, assembly is much more error prone than even C.
\index{label|)}

\subsection{Input and Output \index{I/O|(}}

Input and output are very system dependent activities. It involves
interfacing with the system's hardware. High level languages, like C,
provide standard libraries of routines that provide a simple, uniform
programming interface for I/O.  Assembly languages provide no standard
libraries. They must either directly access hardware (which is a privileged
operation in protected mode) or use whatever low level routines that the
operating system provides.


It is very common for assembly routines to be interfaced with C. One
advantage of this is that the assembly code can use the standard C
library I/O routines.  However, one must know the rules for passing
information between routines that C uses. These rules are too
complicated to cover here. 

To start with, you will be creating your own assembly interpreter so you can
debug with a language you are already familiar with.  Then we'll switch over
to "real" assembly and you'll write your own output procedures for debugging.
\index{I/O|)}

\section{Creating a Program}

Today, it is unusual to create a stand alone program written
completely in assembly language. Assembly is usually used to key certain
critical routines. Why? It is \emph{much} easier to program in a higher level
language than in assembly. Also, using assembly makes a program very hard to
port to other platforms. In fact, it is rare to use assembly at all.

So, why should anyone learn assembly at all?
\begin{enumerate}
\item Sometimes code written in assembly can be faster and smaller than
      compiler generated code.
\item Assembly allows access to direct hardware features of the system that
      might be difficult or impossible to use from a higher level language.
\item Learning to program in assembly helps one gain a deeper understanding of
      how computers work.
\item Learning to program in assembly helps one understand better how compilers
      and high level languages like C work.
\end{enumerate}
These last two points demonstrate that learning assembly can be useful even if
one never programs in it later. In fact, the author rarely programs in
assembly, but he uses the ideas he learned from it everyday.

\subsection{First program}

As is traditional, here is "Hello, world!" in assembly.  We will learn about interrupts ({\code int})
later, but for now just know that you can now add assembly to your resume.

%\begin{lstlisting}[language={[x86masm]Assembler}]
%title Hello World Program         (hello.asm)

%;This program displays "Hello, world!"
%.model small ;small 64k 16 bit
%.stack 100h ;reserves 256 bytes for the stack
%.data ;start definition of variables
%message db "Hello, world!",0dh,0ah,'$' ;message as character array

%.code ;start code portion of program
%main proc ;start of the main procedure
%    mov  ax,@data ;load the address of the data segment into ax
%    mov  ds,ax ;load the address of the data segment into ds

%    mov  ah,9 ;setting for outputting a string
%    mov  dx,offset message ;point dx to the string
 %   int  21h ;output the string

%    mov  ax,4C00h ;exit (ah) with code 0 (al)
 %   int  21h ;exit
%main endp ;end procedure

%end main ;end program
%\end{lstlisting}


\lstinputlisting{../code/hello.asm}




%\begin{figure}
%\centering
%  \includegraphics[width=3in]{hello_world.jpg} %image not OER
%  \caption{Hello World.}
%  \label{fig:helloworld}
%\end{figure}


Line~6 of the program defines a section of the program that specifies
memory to be stored in the data segment (whose name is {\code
.data})\index{data segment}. On line 7, a string is declared. It
will be printed with the interrupt and so must be terminated with a
\$ character.

The code segment \index{code segment} is named {\code .code},
it is where instructions are placed.

\subsection{Compiler dependencies}

We will be using the MASM assembler in combination with DOSBox
both of which can be downloaded from the course website later in the semester.

\subsection{Assembling the code}

The first step is to assemble the code. From the command line, type:
\begin{CodeQuote}
ml hello.asm
\end{CodeQuote}

This will assemble and link your program into an executable which will
be named hello.exe.

You can run your program (assuming there were no errors) with:
\begin{CodeQuote}
hello
\end{CodeQuote}

And it should print:
\begin{CodeQuote}
Hello, world!
\end{CodeQuote}
to the console. 


 {\index{directive!global}  %why is this line important?

\subsection{Understanding an assembly listing file \index{listing file|(}}

The {\code /l {\em listing-file}} switch can be used to tell {\code
masm} to create a listing file of a given name. This file shows how
the code was assembled. Here is how line~8 (in the data
segment) appear in the listing file.

%\begin{Verbatim}[xleftmargin=\AsmMargin]
\begin{lstlisting}[language={[x86masm]Assembler},firstnumber=8]
0000 48 65 6C 6C 6F 2C	    message db "Hello, world!",0dh,0ah,'$'
      20 77 6F 72 6C 64
      21 0D 0A 24
 \end{lstlisting}%Verbatim}
The first column in each line is the
offset (in hex) of the data in the segment. The second column shows the
raw hex values that will be stored. In this case the hex data
correspond to ASCII codes. Finally, the text from the source file is
displayed on the line. The offsets listed in the second column are
very likely \emph{not} the true offsets that the data will be placed
at in the complete program.  Each module may define its own labels in
the data segment (and the other segments, too). In the link step, all these data segment label definitions
are combined to form one data segment. The new final offsets are then
computed by the linker.

Here is a small section (lines~15 to 16 of the source file) of the
text segment in the listing file:
\begin{lstlisting}[language={[x86masm]Assembler},firstnumber=15]
0005  B4 09			    mov  ah,9 ;setting for outputting a string
0007  BA 0000 R		 mov  dx,offset message ;point dx to the string
\end{lstlisting}%Verbatim}
The second column shows the machine code generated by the
assembly. Often the complete code for an instruction can not be
computed yet. For example, in line~16 the offset (or address) of
{\code message} is not known until the code is linked. The assembler
can compute the op-code for the {\code mov} instruction (which from
the listing is BA), but it doesn't write the offset
because the exact value can not be computed yet. In this case, it lists R
after to signify that the linker must resolve the address.  When the code is linked,
the linker will insert the correct offset into the position. Other
instructions, like line~15, do not reference any labels. Here the assembler
can compute the complete machine code.  Here are the same two lines after linking:
\begin{lstlisting}[language={[x86masm]Assembler},firstnumber=15]
05EF:0005  B4 09             mov  ah,9 
05EF:0007  BA 0200	        mov  dx,0002
\end{lstlisting}
Notice how the machine code for line 15 is the same, but on line 16 {\code offset message} is replaced by the offset of the message within the data segment, namely {\code 0002}.  Also the {\code 0000 R} has been replaced by {\code 0200} which is the little endian representation of {\code 0002} (see the next section for a description of little endian representation).  
\index{listing file|)}

\subsubsection{Big and Little Endian Representation \index{endianess|(}}
Different
processors store multibyte integers in different orders in
memory. There are two popular methods of storing integers: \emph{big
endian} and \emph{little endian}.  Big endian is the method that seems the most
natural. The biggest (\emph{i.e.} most significant) byte is stored
first, then the next biggest, \emph{etc.} For example, the dword
00000004 would be stored as the four bytes 00~00~00~04.  IBM
mainframes, most RISC processors and Motorola processors all use this
big endian method. However, Intel-based processors use the little
endian method! Here the least significant byte is stored first. So,
00000004 is stored in memory as 04~00~00~00. This format is hardwired
into the CPU and can not be changed. Normally, the programmer does not
need to worry about which format is used. However, there are
circumstances where it is important.
\begin{enumerate}
\item When binary data is transferred between different computers (either from
      files or through a network).
\item When binary data is written out to memory as a multibyte integer and
      then read back as individual bytes or \emph{vice versa}.
\end{enumerate}

Endianness does not apply to the order of array elements. The first
element of an array is always at the lowest address. This applies to
strings (which are just character arrays). Endianness still
applies to the individual elements of the arrays.
\index{endianess|)}

\begin{figure}[t]
\lstinputlisting{../code/skel.asm}
\caption{Skeleton Program \label{fig:skel}}
\end{figure}

\section{Skeleton File \index{skeleton file}}

Figure~\ref{fig:skel} shows a skeleton file that can be used as a starting
point for writing assembly programs.
 %intro
\include{pcasm2} %basic operations
\include{pcasm3} %bit operations
%-*- latex -*-
\chapter{DOSBox and MASM}
\section {Getting Your Computer Ready}
We will be using the Microsoft Assembler (MASM) which can be downloaded from Canvas.  Unzip it where you'd like to work on your projects and make a note of the absolute address of the file (\verb|C:\Users\etc|)
\index{installing DOSBox|(}
\subsection{Macs}
If you have OSX, I recommend downloading the .dmg file from the \href{https://www.dosbox.com/download.php?main=1}{official DOSBox Downloads page}.  You can install it as you would install a normal .dmg.  \\
\indent After installation hit 'command' + 'shift' + '.' to show hidden files.  Navigate to the hidden {\code Library} folder in your {\code Users/username/} folder.  In there open Preferences and scroll down to a file called 'DOSBox x.xx-x-x Preferences' and open it as text file (the x's are the version of DOSBox you downloaded).\\
Scroll to the bottom and add:
\begin{verbatim}
mount c /Users/yourname/MASM611
set PATH=%PATH%;C:\BIN
c:
\end{verbatim}
Don't forget to replace \verb|/Users/yourname/MASM611| with where you unzipped the MASM folder you downloaded.  

\subsection{PCs}
If you have a PC, I recommend downloading the .exe file from the \href{https://www.dosbox.com/download.php?main=1}{official DOSBox Downloads page}.  You can install it as you would install a normal .exe.  \\
After installation you can go to the Start Menu and select the DosBox folder and then DOSBox Options.  This will open a text file, scroll to the bottom and add: 
\begin{verbatim}
mount c C:\Users\yourname\MASM611
set PATH=%PATH%;C:\BIN
c:
\end{verbatim}
Don't forget to replace \verb|C:\Users\yourname\MASM611| with where you unzipped the MASM folder you downloaded.  \\
If you don't have administrative priviledges on your computer there is a \href{https://portableapps.com/apps/games/dosbox_portable}{Portable Apps version}. 
\index{installing DOSBox|)}
\subsection{Testing DOSBox}
Open DOSBox and test it out by typing {\code ml} at the prompt and then {\code <enter>}, it should say something like:
\begin{verbatim}
Microsoft (R) Macro Assembler Version 6.11
Copyright (C) Microsoft Corp 1981-1993. All rights reserved.

usage: ML [ options ] filelist [ /link linkoptions]
Run "ML /help" or "ML /?" for more info
\end{verbatim}
Congratulations you have successfully installed DOSBox and MASM!  One final note, since it is DOS, the preferred number of characters of a filename is 8.  While it is possible to have longer names, it will make your life easier if all of your assembly program filenames are less than or equal to eight characters (eg {\code filename.asm} is fine). 

\section{Using MASM}
\subsection{Assembling}
First download "hello.asm" from Canvas and put it in your MASM directory.  At the prompt in DOSBox, type in {\code ml hello.asm} and then {\code <enter>}, you should see something like this:
%TODO once this is up on github link to hello.asm
\begin{verbatim}
 Assembling: hello.asm

Microsoft (R) Segmented Executable Linker  Version 5.31.009 Jul 13 1992
Copyright (C) Microsoft Corp 1984-1992.  All rights reserved.

Object Modules [.obj]: hello.obj 
Run File [hello.exe]: "hello.exe"
List File [nul.map]: NUL
Libraries [.lib]: 
Definitions File [nul.def]: 
\end{verbatim}
This is where the assembler would tell you if and where there are errors.  \\
\\
To run the program simply type in the file name without the .asm, in this case it is {\code hello} then {\code <enter>}. \\
\begin{verbatim}
Hello, world!

\end{verbatim}
Hooray!
\section{Debugging}
\subsection{Assembler Errors}
Just like when you compile a program in other languages, the assembler will give you cryptic messages to try to help you debug the errors.  First see if you can spot the errors in this code snippet:
\begin{lstlisting}[language={[x86masm]Assembler},firstnumber=18]
    mov ax,tooLarge    ; ax is 16 bits, tooLarge is 32 bits
    mov eax,tooSmall ; eax is 32 bits, tooSmall is 16 bits
    mov ecx,ax 	     ; ecx is 32 bits, ax is 16 bits
    mov test1,test2    ; test1 is 32 bits, test2 is 32 bits
\end{lstlisting}
When we try to assemble the code this is what we get:
\begin{verbatim}
 Assembling: errors.asm
errors.asm(21): error A2070: invalid instruction operands
errors.asm(18): error A2022: instruction operands must be the same size
errors.asm(19): error A2022: instruction operands must be the same size
errors.asm(20): error A2022: instruction operands must be the same size
\end{verbatim}
On line 21 there are invalid instruction operands, remember memory to memory operations are not ok.  That is usually what this error means.  To fix this you'll have to move test2 into a register first: 
\begin{lstlisting}[language={[x86masm]Assembler},firstnumber=18]
    mov ax,tooLarge    ; ax is 16 bits, tooLarge is 32 bits
    mov eax,tooSmall ; eax is 32 bits, tooSmall is 16 bits
    mov ecx,ax 	     ; ecx is 32 bits, ax is 16 bits
    mov ebx,test2      ; store test2 temporarily into ebx
    mov test1,ebx      ; now move it into test1
\end{lstlisting}
Then lines 18-20 have all different kinds of size problems with operands.  For line 20, you can't move a smaller register into a larger register, if you want to do that you need to use MOVSX or MOVZX.  Helpful comments aside, you don't really know why 18 and 19 have errors until you look at the data segment:
\begin{lstlisting}[language={[x86masm]Assembler},firstnumber=6]
.data ; start definition of variables
	tooLarge dd 6 ; 32 bis
	tooSmall dw 5 ; 16 bits
	test1 dd 492   ; 32 bits
	test2 dd 5678 ; 32 bits
\end{lstlisting}
To fix these errors we could change the size of tooLarge to dw and tooSmall to dd, when we do that they will no longer be the wrong size.  \\
Here is the error free code:
\lstinputlisting[numbers=none]{../code/noerrors.asm}
\subsection{Logic Errors}
Some of the hardest errors you'll find are logic errors.  Generally, the two ways I recommend finding them are print statements and stepping through your code either by hand or with an IDE.  The problem is that we don't know how to do print statements in assembly (yet), nor do we have an IDE.  Luckily there is a program called CodeView that comes with MASM that helps with that.  After your program is assembled and linked you will have a .exe file.  That can be run with CodeView to step through your program and see what is happening to the registers.  Note: sometimes this version of CodeView misbehaves with 32 bit registers!  \\
\indent If we take the fixed code from section 4.3.1 and link it using {\code ml /Zi noerrors.asm}, then we can start with CodeView.  Start by typing {\code cv noerrors} and then enter.  
\begin{figure}
  \includegraphics[width=\linewidth]{images/firstCV.jpg}
  \caption{Initial CodeView Screen.}
  \label{fig:cv1}
\end{figure}
A screen like this one (Figure \ref{fig:cv1}) will pop up. 
The next thing we will do is open the Register window (click {\code Windows -> Register}).  Since we have one memory location changing we'll add a Watch (click {\code Data->Add Watch} and type in test1 and hit ok).  
\begin{figure}
  \includegraphics[width=\linewidth]{images/watchCV.jpg}
  \caption{CodeView with Registers and a Watch.}
  \label{fig:cv2}
\end{figure}
The window now should look like this (Figure \ref{fig:cv2}). 

Now we can step through our code and see how the registers change, hit F10 to step through each line.  When the program is through, it has the final values of the registers, a garbage value for test1 since the program is over, and gives a peek at the assembled code (Figure \ref{fig:cv3}).  
\begin{figure}
  \includegraphics[width=\linewidth]{images/endCV.jpg}
  \caption{CodeView after program runs.}
  \label{fig:cv3}
\end{figure}
Just like in most IDEs you are used to you can also set break points instead of running every line and once you have procedures you can use F8 to trace the procedure or F10 to step over it.  
% do fib for lecture and debug it on the fly do errors for this to show registers and memory
%
\subsection{DOSBox/MASM Quirks}
\begin{itemize}
\item Sometimes your mouse will get stuck in the DOSBox window if you hit alt-Tab that will switch to another open window on your computer and will free your mouse. 
\item If you have an infinite loop or otherwise want to force your program to close, the best way is to just close DOSBox altogether and restart. 
\item In CodeView sometimes it will stop recognizing the mouse, but the keyboard will still work so you can use the shortcut keys.  If you need to choose something from the Menu you can type Alt-F to open the File Menu then you can use the arrow keys and then enter to select what you need.  To exit you can simply type Alt-F4.    
%more?
\end{itemize} %masm
%-*- latex -*-
\chapter{Subprograms}

This chapter looks at using subprograms to make modular programs and to
interface with high level languages (like C). Functions and procedures are
high level language examples of subprograms.

The code that calls a subprogram and the subprogram itself must agree
on how data will be passed between them. These rules on how data will
be passed are called \emph{calling conventions}. \index{calling
convention} A large part of this chapter will deal with the standard C
calling conventions that can be used to interface assembly subprograms
with C programs. This (and other conventions) often pass the addresses
of data (\emph{i.e.} pointers) to allow the subprogram to access the
data in memory.

\section{Indirect Addressing\index{indirect addressing|(}}

Indirect addressing allows registers to act like pointer variables. To
indicate that a register is to be used indirectly as a pointer, it is
enclosed in square brackets ({\code []}). For example:
%tested in subproc0.asm
\begin{lstlisting}[language={[x86masm]Assembler}]
      mov    ax, data     ; normal direct memory addressing of a word
      mov    ebx, offset data      ; ebx = & data
      mov    ax, [ebx]      ; ax = *ebx
\end{lstlisting}
\MarginNote{Note: Lines 1 and 3 perform the same operation.}
Because AX holds a word, line~3 reads a word starting at the address stored 
in EBX.  What EBX is assumed to point to is completely determined by what
instructions are used. Furthermore, even the fact that EBX is a pointer is
completely determined by the what instructions are used. If EBX is used
incorrectly, often there will be no assembler error; however, the program
will not work correctly. This is one of the many reasons that assembly
programming is more error prone than high level programming.

All the 32-bit general purpose (EAX, EBX, ECX, EDX) and index (ESI, EDI)
registers can be used for indirect addressing. 
\index{indirect addressing|)}

\section{Simple Subprogram Example\index{subprogram|(}}

A subprogram is an independent unit of code that can be used from different
parts of a program. In other words, a subprogram is like a function in C. A
jump can be used to invoke the subprogram, but returning presents a problem.
If the subprogram is to be used by different parts of the program, it must
return back to the section of code that invoked it. Thus, the jump back from
the subprogram can not be hard coded to a label. The code below shows how this
could be done using the indirect form of the {\code JMP} instruction. This 
form of the instruction uses the value of a register to determine where to
jump to (thus, the register acts much like a \emph{function pointer} in C.)
Here is a program that uses a subprogram to get a digit from the user.
\lstinputlisting{../code/ch5/subproc1.asm}

The {\code get\_digit} subprogram uses a simple, register-based calling
convention. It expects the EBX register to hold the address of the
BYTE to store the digit input into and the ECX register to hold the
code address of the instruction to jump back to. In line~22,
the {\code ret1} label is used to compute this return address. The {\code \$} 
operator could have been used to compute the return
address. The {\code \$} operator returns the current address for the
line it appears on. The expression {\code \$ + 8} computes the address
of the {\code ret1} label on line~24.

Both of these return address computations are awkward. The first method
requires a label to be defined for each subprogram call. The second method
does not require a label, but does require careful thought. If a near jump
was used instead of a short jump, the number to add to {\code \$} would not
be 8! Fortunately, there is a much simpler way to invoke subprograms. This
method uses the \emph{stack}.

\section{The Stack\index{stack|(}}

Many CPUs have built-in support for a stack. A stack is a Last-In First-Out
(\emph{LIFO}) list. The stack is an area of memory that is organized in this
fashion. The {\code PUSH} instruction adds data to the stack and the
{\code POP} instruction removes data. The data removed is always the last
data added (that is why it is called a last-in first-out list).

The SS segment register specifies the segment that contains the stack (usually
this is the same segment data is stored into). The ESP register contains the
address of the data that would be removed from the stack. This data is said
to be at the \emph{top} of the stack. Data can only be added in double word
units. That is, one can not push a single byte on the stack.

The {\code PUSH} instruction inserts a double word\footnote{Actually
words can be pushed too, but in 32-bit protected mode, it is easier to
work with only double words on the stack.} on the stack by subtracting
4 from ESP and then stores the double word at {\code [ESP]}. The
{\code POP} instruction reads the double word at {\code [ESP]} and
then adds 4 to ESP. The code below demonstrates how these instructions
work and assumes that ESP is initially {\code 0100H}.
%tested in stack.asm
\begin{lstlisting}[language={[x86masm]Assembler}]
    push   dword ptr 1   ; 1 stored at 00FCh, ESP = 00FCh
    push   dword ptr 2   ; 2 stored at 00F8h, ESP = 00F8h
    push   dword ptr 3   ; 3 stored at 00F4h, ESP = 00F4h
    pop    eax        ; EAX = 3, ESP = 00F8h
    pop    ebx        ; EBX = 2, ESP = 00FCh
    pop    ecx        ; ECX = 1, ESP = 0100h
\end{lstlisting}

The stack can be used as a convenient place to store data temporarily. It is
also used for making subprogram calls, passing parameters and local
variables.

The 80x86 also provides a {\code PUSHA} instruction (push all) that pushes the values
of EAX, EBX, ECX, EDX, ESI, EDI and EBP registers (not in this order). The
{\code POPA} instruction (pop all) can be used to pop them all back off.
\index{stack|)}

\section{The CALL and RET Instructions\index{subprogram!calling|(}}
\index{CALL|(}
\index{RET|(}
The 80x86 provides two instructions that use the stack to make calling
subprograms quick and easy. The CALL instruction makes an
unconditional jump to a subprogram and \emph{pushes} the address of
the next instruction on the stack. The RET instruction
\emph{pops off} an address and jumps to that address. When using these
instructions, it is very important that one manage the stack correctly
so that the right number is popped off by the RET instruction!

The previous program can be rewritten to use these new instructions by 
changing lines~21 to 24 to be:
%tested in subproc2.asm
\begin{lstlisting}[language={[x86masm]Assembler}, firstnumber=21]
      mov    ebx, offset input
      ; don't need this anymore
      call   get_digit
      ; this either
\end{lstlisting}
and change the subprogram {\code get\_digit} to:
\begin{lstlisting}[language={[x86masm]Assembler}]
get_digit:
    mov ah, 1
    int 21h
    and al, 0fh     ; char to int
    mov [ebx], al   ; store input into memory
    ret         ; jump back to caller
\end{lstlisting}

There are several advantages to CALL and RET:
\begin{itemize}
\item It is simpler!
\item It allows subprograms calls to be nested easily. Notice that
{\code get\_digit} could call {\code read\_char}. This call pushes another address
on the stack. At the end of {\code read\_char}'s code is a RET that pops
off the return address and jumps back to {\code get\_digit}'s code. Then when
{\code get\_digit}'s RET is executed, it pops off the return address that 
jumps back to {\code main}. This works correctly because of the LIFO
property of the stack.
%tested in subproc2.asm
\begin{lstlisting}[language={[x86masm]Assembler}]
get_digit:
    call read_char
    and al, 0fh     ; char to int
    mov [ebx], al   ; store input into memory
    ret         ; jump back to caller

read_char:
    mov ah, 1
    int 21h
    ret
\end{lstlisting}
\end{itemize}

Remember it is \emph{very} important to pop off all data that is pushed
on the stack. For example, consider the following:
%tested in subproc2.asm
\begin{lstlisting}[language={[x86masm]Assembler}]
get_digit:
    mov ah, 1
    int 21h
    and al, 0fh     ; char to int
    mov [ebx], al   ; store input into memory
    push eax
    ret         ; pops off eax value, not the return address!! 
\end{lstlisting}
This code would not return correctly!
\index{RET|)}
\index{CALL|)}

\section{Calling Conventions\index{calling convention|(}}

When a subprogram is invoked, the calling code and the subprogram (the
\emph{callee}) must agree on how to pass data between them. High-level
languages have standard ways to pass data known as \emph{calling 
conventions}. For high-level code to interface with assembly language, the
assembly language code must use the same conventions as the high-level
language. The calling conventions can differ from compiler to compiler or
may vary depending on how the code is compiled (\emph{e.g.} if
optimizations are on or not). One universal convention is that the code will
be invoked with a {\code CALL} instruction and return via a {\code RET}.

Calling conventions
allow one to create subprograms that are \emph{reentrant}. A reentrant
subprogram may be called at any point of a program safely (even inside
the subprogram itself).

\subsection{"Passing" parameters using registers}
As we saw above, the easiest way you can "pass" parameters is to use the registers.  
I say pass in quotes because you aren't doing anything extra like you do in C++.  Nevertheless,
you can use the existing values in your subprogram.  The changes
 you make will change the registers back in main, similar to passing by reference. 
 \subsubsection{Passing registers by value}
 Sometimes you might want to change the registers in the procedure, but not have those changes
 reflected in main.  This is similar to a local variable in C++ or passing a parameter by value.  There are two 
 common ways to do this, the first is to use push and pop to keep the registers from changing, 
 here is an example from earlier, but now the value of eax is preserved:
 %tested in subproc3.asm
 \begin{lstlisting}[language={[x86masm]Assembler}]
 get_digit:
    push eax
    mov ah, 1
    int 21h
    and al, 0fh     ; char to int
    mov [ebx], al   ; store input into memory
    pop eax
    ret        
\end{lstlisting}
The second way does the exact same thing, but without push and pop cluttering your code 
(the assembler adds those later).  To preserve the registers this way you use PROC with USES, 
this looks cleaner, and is explicit that EAX is being preserved.  
%tested in subproc3.asm
 \begin{lstlisting}[language={[x86masm]Assembler}]
get_digit proc uses eax
    mov ah, 1
    int 21h
    and al, 0fh     ; char to int
    mov [ebx], al   ; store input into memory
    ret         
get_digit endp
\end{lstlisting}
Note the ENDP at the bottom, this is needed to close the procedure.  We didn't need to do this when
we were using a named location for our procedure, but it becomes necessary when you use PROC.
Since we use this we also have to move {\code MAIN ENDP} above the procedure because we can't 
have nested procedures. 
\subsection{Passing parameters on the stack\index{stack|(}\index{stack!parameters|(}}

Parameters to a subprogram may be passed on the stack. They are pushed onto
the stack before the {\code CALL} instruction. Just as in C, if the
parameter is to be changed by the subprogram, the \emph{address} of the 
data must be passed, not the \emph{value}. If the parameter's size is less
than a double word, it must be converted to a double word before being pushed.

The parameters on the stack are not popped off by the subprogram, instead
they are accessed from the stack itself. Why?
\begin{itemize}
\item Since they have to be pushed on the stack before the {\code CALL}
instruction, the return address would have to be popped off first (and
then pushed back on again).
\item Often the parameters will have to be used in several places in the
subprogram. Usually, they can not be kept in a register for the entire
subprogram and would have to be stored in memory. Leaving them on the
stack keeps a copy of the data in memory that can be accessed at any
point of the subprogram.
\end{itemize}

\begin{figure}
\centering
\begin{tabular}{l|c|}
\cline{2-2}
&  \\ \cline{2-2}
ESP + 4 & Parameter \\ \cline{2-2}
ESP     & Return address \\ \cline{2-2}
 & \\ \cline{2-2}
\end{tabular}
\caption{}
\label{fig:stack1}
\end{figure}
Consider 
%\MarginNote{When using indirect addressing, the 80x86 processor 
%accesses different segments depending on what registers are used in the
%indirect addressing expression. ESP (and EBP) use the stack segment while
%EAX, EBX, ECX and EDX use the data seg\-ment. However, this is usually 
%unimportant for most protected mode programs, because for them the data 
%and stack segments are the same.}
a subprogram that is passed a single parameter on the stack. When
the subprogram is invoked, the stack looks like Figure~\ref{fig:stack1}.
The parameter can be accessed using indirect addressing ({\code [ESP+4]}
\footnote{It is legal to add a constant to a register when using indirect
addressing. More complicated expressions are possible too. This topic is covered
in the next chapter}).
\begin{figure}
\centering
\begin{tabular}{l|c|}
\cline{2-2}
&  \\ \cline{2-2}
ESP + 8 & Parameter \\ \cline{2-2}
ESP + 4 & Return address \\ \cline{2-2}
ESP     & subprogram data \\ \cline{2-2}
\end{tabular}
\caption{}
\label{fig:stack2}
\end{figure}

\begin{figure}[t]
 \begin{lstlisting}[language={[x86masm]Assembler}]
subprogram_label:
      push   ebp           ; save original EBP value on stack
      mov    ebp, esp      ; new EBP = ESP
; subprogram code
      pop    ebp           ; restore original EBP value
      ret
\end{lstlisting}
\caption{General subprogram form \label{fig:subskel1}}
\end{figure}

If the stack is also used inside the subprogram to store data, the
number needed to be added to ESP will change. For example,
Figure~\ref{fig:stack2} shows what the stack looks like if a DWORD is
pushed the stack. Now the parameter is at {\code ESP + 8} not {\code
ESP + 4}. Thus, it can be very error prone to use ESP when referencing
parameters. To solve this problem, the 80386 supplies another register
to use: EBP. This register's only purpose is to reference data on the
stack. The C calling convention mandates that a subprogram first save
the value of EBP on the stack and then set EBP to be equal to ESP.
This allows ESP to change as data is pushed or popped off the stack
without modifying EBP. At the end of the subprogram, the original
value of EBP must be restored (this is why it is saved at the start of
the subprogram.)  Figure~\ref{fig:subskel1} shows the general form of
a subprogram that follows these conventions.

\begin{figure}[t]
\centering
\begin{tabular}{ll|c|}
\cline{3-3}
&  & \\ \cline{3-3}
ESP + 8 & EBP + 8 & Parameter \\ \cline{3-3}
ESP + 4 & EBP + 4 & Return address \\ \cline{3-3}
ESP     & EBP     & saved EBP \\ \cline{3-3}
\end{tabular}
\caption{}
\label{fig:stack3}
\end{figure}


Lines 2 and 3 in Figure~\ref{fig:subskel1} make up the general \emph{prologue}
of a subprogram. Lines 5 and 6 make up the \emph{epilogue}. 
Figure~\ref{fig:stack3} shows what the stack looks like immediately
after the prologue. Now the parameter can be access with {\code [EBP + 8]}
at any place in the subprogram without worrying about what else has
been pushed onto the stack by the subprogram.

After the subprogram is over, the parameters that were pushed on the
stack must be removed. The C calling convention \index{calling
convention!C} specifies that the caller code must do this. Other
conventions are different. For example, the Pascal calling convention
\index{calling convention!Pascal} specifies that the subprogram must
remove the parameters.  (There is another form of the RET \index{RET}
instruction that makes this easy to do.) Some C compilers support this
convention too. 

\begin{figure}[t]
 \begin{lstlisting}[language={[x86masm]Assembler}]
      push   dword ptr 1        ; pass 1 as parameter
      call   funC
      add    esp, 4         ; remove parameter from stack
\end{lstlisting}
\caption{Sample C style subprogram call \label{fig:subcallC}}
\end{figure}

\begin{figure}[t]
 \begin{lstlisting}[language={[x86masm]Assembler}]
      push   dword ptr 1        ; pass 1 as parameter
      call   funPascal
      ...
      funPascal PROC
          ;subprogram code
          ret 4 ;remove parameter from stack
       funPascal ENDP
\end{lstlisting}
\caption{Sample Pascal style subprogram call \label{fig:subcallP}}
\end{figure}

Figure~\ref{fig:subcallC} shows how a subprogram using the C calling
convention would be called. Line~3 removes the parameter from the
stack by directly manipulating the stack pointer. A {\code POP}
instruction could be used to do this also, but would require the
useless result to be stored in a register. Actually, for this
particular case, many compilers would use a {\code POP ECX}
instruction to remove the parameter. The compiler would use a {\code
POP} instead of an {\code ADD} because the {\code ADD} requires more
bytes for the instruction. However, the {\code POP} also changes ECX's
value!  Figure~\ref{fig:subcallP} show how a subprogram using the Pascal
calling convention would remove the parameter from the stack. \\

Next is another example program with two subprograms that use
the C calling conventions discussed above. Line~64 (and other lines)
shows that multiple data and code segments may be declared in a single
source file. They will be combined into single data and code segments
in the linking process. Splitting up the data and code into separate
segments allow the data that a subprogram uses to be defined close by
the code of the subprogram.
\index{stack!parameters|)}

\lstinputlisting{../code/ch5/numSum.asm} %numSum


\subsection{Local variables on the stack\index{stack!local variables|(}}

The stack can be used as a convenient location for local variables. This is
exactly where C stores normal (or \emph{automatic} in C lingo) variables.
Using the stack for variables is important if one wishes subprograms to be
reentrant. A reentrant subprogram will work if it is invoked at any place,
including the subprogram itself. In other words, reentrant subprograms
can be invoked \emph{recursively}. Using the stack for variables also saves
memory. Data not stored on the stack is using memory from the beginning of
the program until the end of the program (C calls these types of variables
\emph{global} or \emph{static}). Data stored on the stack only use memory
when the subprogram they are defined for is active.

\begin{figure}[t]
 \begin{lstlisting}[language={[x86masm]Assembler}]
subprogram_label:
      push   ebp                ; save original EBP value on stack
      mov    ebp, esp           ; new EBP = ESP
      sub    esp, LOCAL_BYTES   ; = # bytes needed by locals
; subprogram code
      mov    esp, ebp           ; deallocate locals
      pop    ebp                ; restore original EBP value
      ret
\end{lstlisting}
\caption{General subprogram form with local variables\label{fig:subskel2}}
\end{figure}

\begin{figure}[t]
\begin{lstlisting}[language=C++,frame=tlrb]{}
void calc_sum( int n, int *sumP) {
  int i, sum = 0;

  for( i=1; i <= n; i++ ) {
    sum += i;
   }
  *sumP = sum;
}
\end{lstlisting}
\caption{C version of sum \label{fig:Csum}}
\end{figure}

%this is in subproc4.asm
\begin{figure}[t]
 \begin{lstlisting}[language={[x86masm]Assembler}]
; stack has the address of sumP and the value of n
cal_sum:
      push   ebp
      mov    ebp, esp
      sub    esp, 4               ; make room for local sum
      
      mov    dword ptr [ebp - 4], 0   ; sum = 0
      mov    ebx, 1               ; ebx (i) = 1
for_loop:
      cmp    ebx, [ebp+8]         ; is i <= n?
      jnle   end_for

      add    [ebp-4], ebx         ; sum += i
      inc    ebx
      jmp    short for_loop

end_for:
      mov    ebx, [ebp+12]        ; ebx = sumP
      mov    eax, [ebp-4]         ; eax = sum
      mov    [ebx], eax           ; *sumP = sum;

      mov    esp, ebp
      pop    ebp
      ret
 \end{lstlisting}
\caption{Assembly version of sum\label{fig:Asmsum}}
\end{figure}

Local variables are stored right after the saved EBP value in the stack.
They are allocated by subtracting the number of bytes required from ESP
in the prologue of the subprogram. Figure~\ref{fig:subskel2} shows the 
new subprogram skeleton. The EBP register is used to access local variables.
Consider the C function in Figure~\ref{fig:Csum}. Figure~\ref{fig:Asmsum}
shows how the equivalent subprogram could be written in assembly.

\begin{figure}[t]
\centering
\begin{tabular}{ll|c|}
\cline{3-3}
ESP + 16 & EBP + 12 & address of {\code sumP} \\ \cline{3-3}
ESP + 12 & EBP + 8  & {\code n} \\ \cline{3-3}
ESP + 8  & EBP + 4  & Return address \\ \cline{3-3}
ESP + 4  & EBP      & saved EBP \\ \cline{3-3}
ESP      & EBP - 4  & {\code sum} \\ \cline{3-3}
\end{tabular}
\caption{}
\label{fig:SumStack}
\end{figure}

Figure~\ref{fig:SumStack} shows what the stack looks like after the
prologue of the program in Figure~\ref{fig:Asmsum}. This section of
the stack that contains the parameters, return information and local
variable storage is called a \emph{stack frame}. Every invocation of
a C function creates a new stack frame on the stack.

\begin{figure}[t]
 \begin{lstlisting}[language={[x86masm]Assembler}]
subprogram_label:
      enter  LOCAL_BYTES, 0     ; = # bytes needed by locals
; subprogram code
      leave
      ret
 \end{lstlisting}
\caption{General subprogram form with local variables using 
{\code ENTER} and {\code LEAVE}\label{fig:subskel3}}
\end{figure}

%\MarginNote{Despite the fact that {\code ENTER} and {\code LEAVE} simplify
%the prologue and epilogue they are not used very often. Why? Because
%they are slower than the equivalent simpler instructions! This is an
%example of when one can not assume that a one instruction sequence is
%faster than a multiple instruction one.} 
The prologue and epilogue of a subprogram can be simplified by using
two special instructions that are designed specifically for this
purpose. The {\code ENTER} instruction performs the prologue code and the
{\code LEAVE} performs the epilogue. The {\code ENTER} instruction
takes two immediate operands. For the C calling convention, the second
operand is always 0. The first operand is the number of bytes needed by
local variables. The {\code LEAVE} instruction has no
operands. Figure~\ref{fig:factorial} shows how these instructions are
used. 

\index{stack!local variables|)}
\index{stack|)}
\index{calling convention|)}
\index{subprogram!calling|)}
\section{Multi-Module Programs\index{multi-module programs|(}}

A \emph{multi-module program} is one composed of more than one object
file.  The numSum.asm example program above is a multi-module
program. It consists of the Assembly driver object file and the assembly
object file. Recall that the linker
combines the object files into a single executable program. The linker
must match up references made to each label in one module (\emph{i.e.}
object file) to its definition in another module. In order for module
A to use a label defined in module B, the {\code extern} directive
must be used. After the {\code extern} \index{directive!extern}
directive comes a label and then {\code :proc} to let it know the label is a procedure. 
The directive tells the assembler to treat these labels as \emph{external} to the
module. That is, these are labels that can be used in this module, but
are defined in another. The {\code printInc.lib} file defines the
{\code printDec}, \emph{etc.} routines as external.

\index{multi-module programs|)}

\section{Reentrant and Recursive Subprograms\index{recursion|(}}

\index{subprogram!reentrant|(}
A reentrant subprogram must satisfy the following properties:
\begin{itemize}
\item It must not modify any code instructions. In a high level language
this would be difficult, but in assembly it is not hard for a program to
try to modify its own code. For example:
%this is tested in subproc5.asm
 \begin{lstlisting}[language={[x86masm]Assembler},numbers=none]
  mov word ptr [cs:$+9], 5    ; copy 5 into the word 7 bytes ahead
  add ax, 2               ; previous statement changes 2 to 5!
\end{lstlisting}
This code would work in real mode, but in protected mode operating systems 
the code segment is marked as read only. When the first line above executes,
the program will be aborted on these systems. This type of programming is
bad for many reasons. It is confusing, hard to maintain and does not allow
code sharing (see below).

\item It must not modify global data (such as data in the {\code data} and
the {\code bss} segments). All variables are stored on the stack.

\end{itemize}

There are several advantages to writing reentrant code.
\begin{itemize}
\item A reentrant subprogram can be called recursively.
\item A reentrant program can be shared by multiple processes. On many
multi-tasking operating systems, if there are multiple instances of a
program running, only \emph{one} copy of the code is in memory. Shared
libraries and DLL's (\emph{Dynamic Link Libraries}) use this idea as well.
\item Reentrant subprograms work much better in \emph{multi-threaded}
\footnote{A multi-threaded program has multiple threads of execution. That
is, the program itself is multi-tasked.} pro\-grams. Windows 9x/NT and most
UNIX-like operating systems (Solaris, Linux, \emph{etc.}) support 
multi-threaded programs.
\end{itemize}
\index{subprogram!reentrant|)}

\subsection{Recursive subprograms}

These types of subprograms call themselves. The recursion can be either
\emph{direct} or \emph{indirect}. Direct recursion occurs when a subprogram,
say {\code foo}, calls itself inside {\code foo}'s body. Indirect recursion
occurs when a subprogram is not called by itself directly, but by another
subprogram it calls. For example, subprogram {\code foo} could call
{\code bar} and {\code bar} could call {\code foo}.

Recursive subprograms must have a \emph{termination condition}. When
this condition is true, no more recursive calls are made. If a
recursive routine does not have a termination condition or the condition
never becomes true, the recursion will never end (much like an infinite
loop).

%also in subproc5.asm
\begin{figure}
 \begin{lstlisting}[language={[x86masm]Assembler}]
; finds n!
factorial proc
    enter 0,0 ; sets up stack frame
    mov    eax, [ebp+4] ; eax = n retrieve parameter from stack
    cmp    eax, 1 ; check termination condition
    jbe    term_cond    ; if n <= 1, terminate
    dec    eax    ; n--
    push   eax    ; call with (n-1)
    call   factorial ; eax = fact(n-1)
    mul    dword ptr [ebp+4]   ; edx:eax = eax * [ebp+4]
    term_cond:
    leave ; terminates stack frame
    ret 1 ; clean up the eax we pushed earlier
factorial endp
\end{lstlisting}
\caption{Recursive factorial function\label{fig:factorial}}
\end{figure}

\begin{figure}
\centering
%\includegraphics{factStack.eps}
\input{factStack.latex}
\caption{Stack frames for factorial function\label{fig:factStack}}
\end{figure}

Figure~\ref{fig:factorial} shows a function that calculates factorials
recursively. It could be called from C with:
\begin{lstlisting}[stepnumber=0]{}
x = factorial(3);         /* find 3! */
\end{lstlisting}
Figure~\ref{fig:factStack} shows what the stack looks like at its deepest
point for the above function call.


%Figures~\ref{fig:rec2C} and \ref{fig:rec2Asm} show another more
%complicated recursive example in C and assembly, respectively. What is
%the output is for {\code factorial(3)}? Note that the {\code ENTER} instruction
%creates a new {\code i} on the stack for each recursive call. Thus, each
%recursive instance of {\code f} has its own independent variable {\code i}.
%Defining {\code i} as a double word in the {\code data} segment would not
%work the same. 
\index{recursion|)}

\index{subprogram|)}
 %procs 
% -*-latex-*-
\chapter{Arrays}
\index{arrays|(}
\section{Introduction}

An \emph{array} is a contiguous block of data in memory. Each element
of the list must be the same type and use exactly the same number of bytes
of memory for storage. Because of these properties, arrays allow efficient
access of the data by its position (or index) in the array. The address
of any element can be computed by knowing three facts:
\begin{itemize}
\item The address of the first element of the array.
\item The number of bytes in each element
\item The index of the element
\end{itemize}

It is convenient to consider the index of the first element of the array
to be zero (just as in C). It is possible to use other values for the
first index, but it complicates the computations.

\subsection{Defining arrays\index{arrays!defining|(}}

\begin{figure}[t]
%tested in array1.asm
\begin{lstlisting}[language={[x86masm]Assembler}]
.data
	; define array of 10 double words initialized to 1,2,..,10
	a1 dd 1, 2, 3, 4, 5, 6, 7, 8, 9, 10
	; define array of 10 words initialized to 0
	a2 dw  0, 0, 0, 0, 0, 0, 0, 0, 0, 0
	; same as before using dup
	a3 dw 10 dup (0)
	; define array of bytes with 200 0's and then 100 1's
	a4 db 200 dup (0)
	   db 100 dup (1)
	; define an array of 10 uninitialized double words
	a5 dd 10 dup (?)
	; define an array of 100 uninitialized words
	a6 dw 100 dup (?)
\end{lstlisting}
\caption{Defining arrays\label{fig:DataArrays}}
\end{figure}

\subsubsection{Defining arrays in the {\code data} segment
               \index{arrays!defining!static}}

To define an initialized array in the {\code data} segment, use the
normal {\code db}, {\code dw}, \emph{etc.} 
\index{directive!D\emph{X}}directives. MASM also provides a useful directive
named {\code DUP} \index{directive!DUP} that can be used to repeat a statement many times
without having to duplicate the statements by hand.
Figure~\ref{fig:DataArrays} shows several examples of these.

To define an uninitialized array, use the
{\code ?} \index{directive!?}
directive. Figure~\ref{fig:DataArrays} also shows examples of these
types of definitions.

\begin{figure}[t]
\centering
\begin{tabular}{l|c|ll|c|}
\cline{2-2} \cline{5-5}
EBP - 1  & char    & \hspace{2em} &           & \\
\cline{2-2}
         & unused  &              &           & \\
\cline{2-2}
EBP - 8  & dword 1 &              &           & \\
\cline{2-2}
EBP - 12 & dword 2 &              &           & word \\
\cline{2-2}
         &         &              &           & array \\
         &         &              &           & \\
         & word    &              &           & \\
         & array   &              & EBP - 100 & \\
\cline{5-5}
         &         &              & EBP - 104 & dword 1 \\
\cline{5-5}
         &         &              & EBP - 108 & dword 2 \\
\cline{5-5}
         &         &              & EBP - 109 & char \\
\cline{5-5}
EBP - 112 &        &              &           & unused \\
\cline{2-2} \cline{5-5}
\end{tabular}
\caption{Arrangements of the stack\label{fig:StackLayouts}}
\end{figure}

\subsubsection{Defining arrays as local variables on the stack\index{arrays!defining!local variable}}

There is no direct way to define a local array variable on the
stack. As before, one computes the total bytes required by \emph{all}
local variables, including arrays, and subtracts this from ESP (either
directly or using the {\code ENTER} instruction). For example, if a
function needed a character variable, two double word integers and a
50 element word array, one would need $1 + 2 \times 4 + 50 \times 2 =
109$ bytes. However, the number subtracted from ESP should be a
multiple of four (112 in this case) to keep ESP on a double word
boundary. One could arrange the variables inside this 109 bytes in
several ways. Figure~\ref{fig:StackLayouts} shows two possible ways. The
unused part of the first ordering is there to keep the double words on
double word boundaries to speed up memory accesses.
\index{arrays!defining|)}

\subsection{Accessing elements of arrays\index{arrays!accessing|(}}

The {\code [ ]} operator in assembly language is much more versatile than C.  You can
use it similarly to how you might use it in C to access an array, but to
access the correct element of an array, its address must be computed. Consider
the following two array definitions:
%tested in array2.asm
\begin{lstlisting}[language={[x86masm]Assembler}]
array1  db 5, 4, 3, 2, 1     ; array of bytes
array2  dw 5, 4, 3, 2, 1     ; array of words
array3  dd 5, 4, 3, 2, 1     ; array of double words
\end{lstlisting}
Here are some examples using these arrays:
\begin{lstlisting}[language={[x86masm]Assembler}]
	mov    al, [array1]             ; al = array1[0]
	mov    al, array1[1]         ; al = array1[1]
	mov    [array1 + 3], al         ; array1[3] = al
	mov    ax, array2             ; ax = array2[0]
	mov    ax, array2[2]         ; ax = array2[1] (NOT array2[2]!)
	mov    [array2 + 6], ax         ; array2[3] = ax
	mov    ax, [array2 + 1]         ; ax = ??
\end{lstlisting}
In line~5, element 1 of the word array is referenced, not element 2. Why?
Words are two byte units, so to move to the next element of a word array,
one must move two bytes ahead, not one. Line~7 will read one byte from the
first element and one from the second. In C, the compiler looks at the type
of a pointer in determining how many bytes to move in an expression that
uses pointer arithmetic so that the programmer does not have to. However,
in assembly, it is up to the programmer to take the size of array elements
in account when moving from element to element. \index{ArrayOps(}Luckily, there is a way 
to calculate the important values associated with an array, see Table~\ref{tab:ArrayOps}.

\begin{table}[]
\begin{tabular}{ll}
\textbf{Command} & \textbf{Purpose}                    \\
type             & The number of bytes in each element \\
lengthof           & The number of elements in an array  \\
sizeof         & The number of bytes in an array    
\end{tabular}
\caption{Helpful Array Operators\label{tab:ArrayOps}}
\end{table}
\index{ArrayOps)}
%tested in array3.asm way 1
\begin{figure}[t]
\begin{lstlisting}[language={[x86masm]Assembler},frame=single]
	mov    ebx, offset array1           ; ebx = address of array1
	mov    dx, 0                 ; dx will hold sum
	mov    ah, 0                 ; ?
	mov    ecx, 5
lp:
	mov    al, [ebx]             ; al = *ebx
	add    dx, ax                ; dx += ax (not al!)
	inc    ebx                   ; bx++
	loop   lp
\end{lstlisting}
\caption{Summing elements of an array (Version 1)\label{fig:SumArray1}}
\end{figure}
%tested in array3.asm way 2
\begin{figure}[t]
\begin{lstlisting}[language={[x86masm]Assembler},frame=single]
      mov    ebx, offset array1           ; ebx = address of array1
      mov    dx, 0                 ; dx will hold sum
      mov    ecx, 5
lp:
      add    dl, [ebx]             ; dl += *ebx
      jnc    next                  ; if no carry goto next
      inc    dh                    ; inc dh
next:
      inc    ebx                   ; bx++
      loop   lp
\end{lstlisting}
\caption{Summing elements of an array (Version 2)\label{fig:SumArray2}}
\end{figure}
%tested in array3.asm way 3
\begin{figure}[t]
\begin{lstlisting}[language={[x86masm]Assembler},frame=single]
      mov    ebx, offset array1           ; ebx = address of array1
      mov    dx, 0                 ; dx will hold sum
      mov    ecx, 5
lp:
      add    dl, [ebx]             ; dl += *ebx
      adc    dh, 0                 ; dh += carry flag + 0
      inc    ebx                   ; bx++
      loop   lp
\end{lstlisting}
\caption{Summing elements of an array (Version 3)\label{fig:SumArray3}}
\end{figure}

Figure~\ref{fig:SumArray1} shows a code snippet that adds all the
elements of {\code array1} in the previous example code. In
line 7, AX is added to DX. Why not AL? First, the
two operands of the {\code ADD} instruction must be the same
size. Secondly, it would be easy to add up bytes and get a sum that
was too big to fit into a byte. By using DX, sums up to 65,535 are
allowed. However, it is important to realize that AH is being added
also.  This is why AH is set to zero\footnote{Setting AH to zero is
implicitly assuming that AL is an unsigned number. If it is signed,
the appropriate action would be to insert a {\code CBW} instruction
between lines~6 and 7} in line~3.

Figures~\ref{fig:SumArray2} and \ref{fig:SumArray3} show two alternative
ways to calculate the sum. The differences are mostly around lines~6 and 7
of each Figure.

\subsection{More advanced indirect addressing\index{indirect addressing!arrays|(}}

Not surprisingly, indirect addressing is often used with arrays. The most
general form of an indirect memory reference is:
\begin{center}
{\code [ \emph{base reg} + \emph{factor}*\emph{index reg} + 
      \emph{constant}]}
\end{center}
where:
\begin{description}
\item[base reg] is one of the registers EAX, EBX, ECX, EDX, EBP, ESP, ESI
                or EDI.
\item[factor] is either 1, 2, 4 or 8 . (If 1, factor is omitted.)
\item[index reg] is one of the registers EAX, EBX, ECX, EDX, EBP, ESI, EDI.
                 (Note that ESP is not in list.)
\item[constant] is a 32-bit constant. The constant can be a label (or
                a label expression).
\end{description}

%\subsection{Example}
Figure~\ref{fig:SumArray4} is another sum array example that uses the general form. 

%tested in array4.asm
\begin{figure}[t]
\begin{lstlisting}[language={[x86masm]Assembler},frame=single]
	mov    ebx, offset array3     ; ebx = address of array3
	mov    edx, 0                 ; edx will hold sum
	mov    ecx, 5                 ; ecx is our loop counter
lp:            ;[baseReg + factor*indexReg + constant]
	add    edx, [ebx+(type array3)*ecx-(type array3)]; edx += array3[ecx-1]
	loop   lp
\end{lstlisting}
\caption{Summing elements of an array (Version 4)\label{fig:SumArray4}}
\end{figure}

\index{indirect addressing!arrays|)}
\index{arrays!accessing|)}

\subsubsection{The {\code LEA} instruction\index{LEA|(}}

The {\code LEA} instruction is used to calculate addresses instead of 
{\code offset}. In addition to this, you could use it for other purposes. A fairly
common one is for fast computations. Consider the following:
%tested in array4.asm
\begin{lstlisting}[language={[x86masm]Assembler},numbers=none]
      lea    ebx, [4*eax + eax]
\end{lstlisting}
This effectively stores the value of $5 \times \mathtt{EAX}$ into EBX.
Using {\code LEA} to do this is both easier and faster than using
{\code MUL}\index{MUL}. However, one must realize that the expression
inside the square brackets \emph{must} be a legal indirect address.
Thus, for example, this instruction can not be used to multiple by 6
quickly.
\index{LEA|)}


\subsection{Multidimensional Arrays\index{arrays!multidimensional|(}}

Multidimensional arrays are not really very different than the plain
one dimensional arrays already discussed. In fact, they are represented 
in memory as just that, a plain one dimensional array.

\subsubsection{Two Dimensional Arrays\index{arrays!multidimensional!two dimensional|(}}
Not surprisingly, the simplest multidimensional array is a two dimensional
one. A two dimensional array is often displayed as a grid of elements. Each
element is identified by a pair of indices. By convention, the first index
is identified with the row of the element and the second index the column.

Consider an array with three rows and two columns defined as: 
\begin{lstlisting}[stepnumber=0]{}
  int a[3][2];
\end{lstlisting}
The C compiler would reserve room for a 6 ($= 2 \times 3$) integer array and
map the elements as follows:

\parbox{\textwidth}{
\vspace{0.5em}
\centering
\begin{tabular}{||l|c|c|c|c|c|c||}
\hline
Index & 0 & 1 & 2 & 3 & 4 & 5 \\
\hline
Element & a[0][0] & a[0][1] & a[1][0] & a[1][1] & a[2][0] & a[2][1]  \\
\hline
\end{tabular}
\vspace{0.5em}
}
\noindent What the table attempts to show is that the element referenced as 
{\code a[0][0]} is stored at the beginning of the 6 element one
dimensional array. Element {\code a[0][1]} is stored in the next
position (index~1) and so on. Each row of the two dimensional array is
stored contiguously in memory. The last element of a row is followed
by the first element of the next row. This is known as the
\emph{rowwise} representation of the array and is how a C/C++ compiler would
represent the array.

How does the compiler determine where {\code a[i][j]} appears in the rowwise
representation? A simple formula will compute the index from {\code i} and
{\code j}. The formula in this case is $2i + j$. It's not too hard to see how
this formula is derived. Each row is two elements long; so, the first element
of row $i$ is at position $2i$. Then the position of column $j$ is found by
adding $j$ to $2i$. This analysis also shows how the formula is generalized 
to an array with {\code N} columns: $N \times i + j$. Notice that the formula
does \emph{not} depend on the number of rows.

As an example, let us see how \emph{gcc} compiles the following code (using the
array {\code a} defined above):
\begin{lstlisting}[stepnumber=0]{}
  x = a[i][j];
\end{lstlisting}
The compiler essentially converts the code to:
\begin{lstlisting}[stepnumber=0]{}
  x = *(&a[0][0] + 2*i + j);
\end{lstlisting}
and in fact, the programmer could write this way with the same result.  The example code 
below shows how you might find the sum of a 2D array in assembly.

\lstinputlisting{../code/ch6/2dsum.asm}

There is nothing magical about the choice of the rowwise representation of the
array. A columnwise representation would work just as well: 

\parbox{\textwidth}{
\vspace{0.5em}
\centering
\begin{tabular}{||l|c|c|c|c|c|c||}
\hline
Index & 0 & 1 & 2 & 3 & 4 & 5 \\
\hline
Element & a[0][0] & a[1][0] & a[2][0] & a[0][1] & a[1][1] & a[2][1]  \\
\hline
\end{tabular}
\vspace{0.5em}
}
\noindent In the columnwise representation, each column is stored contiguously. 
Element {\code [i][j]} is stored at position $i + 3j$. Other languages
(FORTRAN, for example) use the columnwise representation. This is
important when interfacing code with multiple languages.
\index{arrays!multidimensional!two dimensional|)}

\subsubsection{Dimensions Above Two}
For dimensions above two, the same basic idea is applied. Consider a three
dimensional array:
\begin{lstlisting}[stepnumber=0]{}
  int b[4][3][2];
\end{lstlisting}
This array would be stored like it was four two dimensional arrays each of size
{\code [3][2]} consecutively in memory. The table below shows how it starts out:

\parbox{\textwidth}{
\vspace{0.5em}
\centering
\begin{tabular}{||l|c|c|c|c|c|c||}
\hline
Index & 0 & 1 & 2 & 3 & 4 & 5  \\
\hline
Element & b[0][0][0] & b[0][0][1]  & b[0][1][0] & b[0][1][1] & b[0][2][0]
&  b[0][2][1]  \\
\hline
\hline
Index & 6 & 7 & 8 & 9 & 10 & 11 \\
\hline
Element & b[1][0][0] & b[1][0][1] & b[1][1][0] & b[1][1][1]  & b[1][2][0] 
& b[1][2][1] \\
\hline
\end{tabular}
\vspace{0.5em}
}
\noindent The formula for computing the position of {\code b[i][j][k]}
is $6i + 2j + k$. The 6 is determined by the size of the {\code
[3][2]} arrays. In general, for an array dimensioned as {\code
a[L][M][N]} the position of element {\code a[i][j][k]} will be $M\times N\times i 
+ N \times j + k$. Notice again that the first
dimension ({\code L}) does not appear in the formula.

For higher dimensions, the same process is generalized. For an $n$ dimensional
array of dimensions $D_1$ to $D_n$, the position of element denoted by the
indices $i_1$ to $i_n$ is given by the formula:
\begin{displaymath}
D_2 \times D_3 \cdots \times D_n \times i_1 + D_3 \times D_4 \cdots \times D_n 
\times i_2 + \cdots + D_n \times i_{n-1} + i_n
\end{displaymath}
or for the \"{u}ber math geek, it can be written more succinctly as:
\begin{displaymath}
\sum_{j=1}^{n} \: \left( \prod_{k=j+1}^{n} D_k \right) \: i_j
\end{displaymath}
\MarginNote{This is where you can tell the author was a physics major. (Or was the
reference to FORTRAN a giveaway?)}
The first dimension, $D_1$, does not appear in the formula.

For the columnwise representation, the general formula would be:
\begin{displaymath}
i_1 + D_1 \times i_2 + \cdots + D_1 \times D_2 \times \cdots \times D_{n-2} 
\times i_{n-1} + D_1 \times D_2 \times \cdots \times D_{n-1} \times i_n
\end{displaymath}
or in \"{u}ber math geek notation:
\begin{displaymath}
\sum_{j=1}^{n} \: \left( \prod_{k=1}^{j-1} D_k \right) \: i_j
\end{displaymath}
In this case, it is the last dimension, $D_n$, that does not appear in the
formula.
%
%\subsubsection{Passing Multidimensional Arrays as Parameters in C\index{arrays!multidimensional!parameters|(}}
%
%The rowwise representation of multidimensional arrays has a direct
%effect in C programming. For one dimensional arrays, the size of the
%array is not required to compute where any specific element is located
%in memory. This is not true for multidimensional arrays.  To access
%the elements of these arrays, the compiler must know all but the first
%dimension. This becomes apparent when considering the prototype of a
%function that takes a multidimensional array as a parameter. The
%following will not compile:
%\begin{lstlisting}[stepnumber=0]{}
%  void f( int a[ ][ ] );  /* no dimension information */
%\end{lstlisting}
%However, the following does compile:
%\begin{lstlisting}[stepnumber=0]{}
%  void f( int a[ ][2] );
%\end{lstlisting}
%Any two dimensional array with two columns can be passed to this function.
%The first dimension is not required\footnote{A size can be specified here,
%but it is ignored by the compiler.}.
%
%Do not be confused by a function with this prototype:
%\begin{lstlisting}[stepnumber=0]{}
%  void f( int * a[ ] );
%\end{lstlisting}
%This defines a single dimensional array of integer pointers (which incidently
%can be used to create an array of arrays that acts much like a two dimensional
%array).
%
%For higher dimensional arrays, all but the first dimension of the array must
%be specified for parameters. For example, a four dimensional array parameter
%might be passed like:
%\begin{lstlisting}[stepnumber=0]{}
%  void f( int a[ ][4][3][2] );
%\end{lstlisting}
%\index{arrays!multidimensional!parameters|)}
\index{arrays!multidimensional|)}

\section{Array/String Instructions}
\index{string instructions|(} 

The 80x86 family of processors provide several instructions that are
designed to work with arrays. These instructions are called
\emph{string instructions}. They use the index registers (ESI and EDI)
to perform an operation and then to automatically increment or
decrement one or both of the index registers. The \emph{direction
flag} (DF) \index{register!FLAGS!DF} in the FLAGS register determines
where the index registers are incremented or decremented. There are
two instructions that modify the direction flag:
\begin{description}
\item[CLD] \index{CLD} clears the direction flag. In this state, the index registers
           are incremented.
\item[STD] \index{STD} sets the direction flag. In this state, the index registers are
           decremented.
\end{description}
A \emph{very} common mistake in 80x86 programming is to forget to explicitly
put the direction flag in the correct state. This often leads to code that
works most of the time (when the direction flag happens to be in the desired
state), but does not work \emph{all} the time.

%tested in string1.asm
\begin{figure}[t]
\centering
{\code
\begin{tabular}{|lp{1.5in}|lp{1.5in}|}
\hline
LODSB & AL = [ESI]\newline ESI = ESI $\pm$ 1 & 
STOSB & [EDI] = AL\newline EDI = EDI $\pm$ 1 \\
\hline
LODSW & AX = [ESI]\newline ESI = ESI $\pm$ 2 & 
STOSW & [EDI] = AX\newline EDI = EDI $\pm$ 2 \\
\hline
LODSD & EAX = [ESI]\newline ESI = ESI $\pm$ 4 & 
STOSD & [EDI] = EAX\newline EDI = EDI $\pm$ 4 \\
\hline
\end{tabular}
}
\caption{Reading and writing string instructions\label{fig:rwString}
         \index{LODSB} \index{STOSB} \index{LODSW} \index{LODSD} \index{STOSD}}
\end{figure}

%tested in string2.asm
\begin{figure}[t]
\begin{lstlisting}[language={[x86masm]Assembler},frame=single]
.data ; start definition of variables
string db "test",0
array1  dd  1, 2, 3, 4, 5, 6, 7, 8, 9, 10
array2  dd 10 dup (?)

.code ; start code portion of program
main proc ; start of the main procedure
    mov  eax,@data ; load the address of the data segment into eax
    mov  ds,eax ; load the address of the data segment into ds
    mov  es,eax ; load the address of the data segment into es
    ; the three previous instructions initalize the data segment and 

    ;copies array1 into array2
      cld      ; don't forget this!
      mov    esi, offset array1
      mov    edi, offset array2
      mov    ecx, lengthof array1
lp:
      lodsd  
      stosd
      loop  lp
\end{lstlisting}
\caption{Load and store example\label{fig:lodEx}}
\end{figure}

\subsection{Reading and writing memory}

The simplest string instructions either read or write memory or
both. They may read or write a byte, word or double word at a time.
Figure~\ref{fig:rwString} shows these instructions with a short
pseudo-code description of what they do. There are several points to
notice here. First, ESI is used for reading and EDI for writing. It is
easy to remember this if one remembers that SI stands for \emph{Source
Index} and DI stands for \emph{Destination
Index}. \index{register!ESI} \index{register!EDI} Next, notice that
the register that holds the data is fixed (either AL, AX or
EAX). Finally, note that the storing instructions use ES to determine
the segment to write to, not DS. In protected mode programming this is
not usually a problem, since there is only one data segment and ES
should be automatically initialized to reference it (just as DS
is). However, in real mode programming, it is \emph{very} important
for the programmer to initialize ES to the correct segment
\index{register!segment} selector value.  What this means to you is that 
you need to add {\code mov es,eax} to the beginning of your code (see 
line~10 in Figure~\ref{fig:lodEx}).  Figure~\ref{fig:lodEx} shows an example use of these
instructions that copies an array into another.

%\footnote{Another complication
%is that one can not copy the value of the DS register into the ES
%register directly using a single {\code MOV} instruction. Instead, the
%value of DS must be copied to a general purpose register (like AX) and
%then copied from that register to ES using two {\code MOV}
%instructions.}. 
%tested in string1.asm
\begin{figure}[t]
\centering
{\code
\begin{tabular}{|lp{2.5in}|}
\hline
MOVSB & byte [EDI] = byte [ESI] \newline ESI = ESI $\pm$ 1 \newline
        EDI = EDI $\pm$ 1 \\
\hline
MOVSW & word [EDI] = word [ESI] \newline ESI = ESI $\pm$ 2 \newline
        EDI = EDI $\pm$ 2 \\
\hline
MOVSD & dword [EDI] = dword [ESI] \newline ESI = ESI $\pm$ 4 \newline
        EDI = EDI $\pm$ 4 \\
\hline
\end{tabular}
}
\caption{Memory move string instructions\label{fig:movString} \index{MOVSB}
         \index{MOVSW} \index{MOVSD}}
\end{figure}

%tested in string1.asm
\begin{figure}[t]
\begin{lstlisting}[language={[x86masm]Assembler},frame=single]
          cld       ; don't forget this!
          mov    edi, offset array1
          mov    ecx, 10
          xor    eax, eax
          rep stosd
\end{lstlisting}
\caption{Zero array example\label{fig:zeroArrayEx}}
\end{figure}

The combination of a {\code LODSx} and {\code STOSx} instruction (as in
lines~19 and 20 of Figure~\ref{fig:lodEx}) is very common. In fact, this
combination can be performed by a single {\code MOVSx} string instruction.
Figure~\ref{fig:movString} describes the operations that these 
instructions perform. Lines~19 and 20 of Figure~\ref{fig:lodEx} could be
replaced with a single {\code MOVSD} instruction with the same effect. The
only difference would be that the EAX register would not be used at all
in the loop. %the previous statement was tested in string2.asm

\subsection{The {\code REP} instruction prefix\index{REP|(}}

The 80x86 family provides a special instruction prefix\footnote{A
instruction prefix is not an instruction, it is a special byte that is
placed before a string instruction that modifies the instructions
behavior. Other prefixes are also used to override segment defaults of
memory accesses} called {\code REP} that can be used with the above string
instructions. This prefix tells the CPU to repeat the next string instruction
a specified number of times. The ECX register is used to count the iterations
(just as for the {\code LOOP} instruction). Using the {\code REP} prefix, 
the loop in Figure~\ref{fig:lodEx} (lines~18 to 21) could be replaced with
a single line:
%tested in string2.asm
\begin{lstlisting}[language={[x86masm]Assembler},frame=none, numbers=none]
      rep movsd
\end{lstlisting}
Figure~\ref{fig:zeroArrayEx} shows another example that zeroes out the
contents of an array.
\index{REP|)}

%tested in string1.asm
\begin{figure}[t]
\centering
{\code
\begin{tabular}{|lp{3.5in}|}
\hline
CMPSB & compares byte [ESI] and byte [EDI] \newline ESI = ESI $\pm$ 1 
        \newline EDI = EDI $\pm$ 1 \\
\hline
CMPSW & compares word [ESI] and word [EDI] \newline ESI = ESI $\pm$ 2 
        \newline EDI = EDI $\pm$ 2 \\
\hline
CMPSD & compares dword [ESI] and dword [EDI] \newline ESI = ESI $\pm$ 4 
        \newline EDI = EDI $\pm$ 4 \\
\hline
SCASB & compares AL and [EDI] \newline EDI $\pm$ 1 \\
\hline
SCASW & compares AX and [EDI] \newline EDI $\pm$ 2 \\
\hline
SCASD & compares EAX and [EDI] \newline EDI $\pm$ 4 \\
\hline
\end{tabular}
}
\caption{Comparison string instructions\label{fig:cmpString}
         \index{CMPSB} \index{CMPSW} \index{CMPSD} \index{SCASB}
         \index{SCASW} \index{SCASD}}
\end{figure}

%tested in string3.asm part 1
\begin{figure}[t]
\begin{lstlisting}[language={[x86masm]Assembler},frame=single]
    cld
    mov    edi, offset array    ; pointer to start of array
    mov    ecx, lengthof array  ; number of elements
    mov    eax, 12       ; number to scan for
lp:
    scasd
    je     found
    loop   lp
    mov dx, offset notFoundStr
    jmp    onward
found:
    sub edi, 4  ; edi now points to 12 in array
    mov dx, offset foundStr
onward:
    mov ah, 9
    int 21h
\end{lstlisting}
\caption{Search example\label{fig:srchStrEx}}
\end{figure}

\subsection{Comparison string instructions}

Figure~\ref{fig:cmpString} shows several new string instructions that
can be used to compare memory with other memory or a register. They
are useful for comparing or searching arrays. They set the FLAGS
register just like the {\code CMP} instruction. The {\code CMPSx}
\index{CMPSB} \index{CMPSW} \index{CMPSD} instructions compare
corresponding memory locations and the {\code SCASx} \index{SCASB}
\index{SCASW} \index{SCASD} scan memory locations for a specific
value.

Figure~\ref{fig:srchStrEx} shows a short code snippet that searches
for the number 12 in a double word array. The {\code SCASD} instruction on
line~6 always adds 4 to EDI, even if the value
searched for is found. Thus, if one wishes to find the address of the 12
found in the array, it is necessary to subtract 4 from EDI (as 
line~12 does).

\begin{figure}[t]
\centering
\begin{tabular}{|l|p{4in}|}
\hline
{\code REPE}, {\code REPZ} & repeats instruction while Z flag is set or
                             at most ECX times \\
\hline
{\code REPNE}, {\code REPNZ} & repeats instruction while Z flag is cleared or
                             at most ECX times \\
\hline
\end{tabular}
\caption{{\code REPx} instruction prefixes\label{fig:repx} \index{REPE} 
          \index{REPNE}}
\end{figure}

%tested in string3.asm part 2
\begin{figure}
\begin{lstlisting}[language={[x86masm]Assembler},frame=single]
    cld
    mov    esi, offset block1 ; address of first block
    mov    edi, offset block2 ; address of second block
    mov    ecx, sizeof block1 ; size of blocks in bytes
    repe   cmpsb              ; repeat while Z flag is set
    je     equal              ; if Z set, blocks equal
    ; code to perform if blocks are not equal
    jmp    onward
equal:
   ; code to perform if equal
onward:
\end{lstlisting}
\caption{Comparing memory blocks\label{fig:cmpBlocksEx}}
\end{figure}

\subsection{The {\code REPx} instruction prefixes}

There are several other {\code REP}-like instruction prefixes that can be
used with the comparison string instructions. Figure~\ref{fig:repx} shows
the two new prefixes and describes their operation. {\code REPE} \index{REPE} and
{\code REPZ} are just synonyms for the same prefix (as are {\code REPNE} \index{REPNE}
and {\code REPNZ}). If the repeated comparison string instruction stops
because of the result of the comparison, the index register or registers
are still incremented and ECX decremented; however, the FLAGS register
still holds the state that terminated the repetition. 
\MarginNote{Why can one not just look to see if ECX is zero after the
repeated comparison?} Thus, it is possible
to use the Z flag to determine if the repeated comparisons stopped because
of a comparison or ECX becoming zero.

Figure~\ref{fig:cmpBlocksEx} shows an example code snippet that determines
if two blocks of memory are equal. The {\code JE} on 
line~6 of the example checks to see the result of the
previous instruction. If the repeated comparison stopped because it found
two unequal bytes, the Z flag will still be cleared and no branch is made;
however, if the comparisons stopped because ECX became zero, the Z flag
will still be set and the code branches to the {\code equal} label.

%\subsection{Example}


\index{string instructions|)}
\index{arrays|)}













 %arrays
% -*-latex-*-
\chapter{Structures}

\section{Structures\index{structures|(}}

\subsection{Introduction}

Structures are used in C to group together related data into a composite 
variable. This technique has several advantages:
\begin{enumerate}
\item It clarifies the code by showing that the data defined in the structure
      are intimately related.
\item It simplifies passing the data to functions. Instead of passing
      multiple variables separately, they can be passed as a single unit.
\item It increases the \index{locality}\emph{locality}\footnote{See the virtual memory 
management section of any Operating System text book for discussion of
this term.} of the code.
\end{enumerate}

From the assembly standpoint, a structure can be considered as an
array with elements of \emph{varying} size. The elements of real
arrays are always the same size and type. This property is what allows
one to calculate the address of any element by knowing the starting
address of the array, the size of the elements and the desired
element's index.

A structure's elements do not have to be the same size (and usually
are not). Because of this each element of a structure must be
explicitly specified and is given a \emph{tag} (or name) instead of a
numerical index.

In assembly, the element of a structure will be accessed in a similar
way as in C.  To access an element, you give the structure variable name
then a dot then the field.  

For example, consider the following structure shown in Figure~\ref{fig:CStructs}.
%tested in struct1.cpp
\begin{figure}[t]
\begin{lstlisting}[language=C,stepnumber=0]{}
struct exampleStruct {
  short int x=0;    /* 2-byte integer */
  int       y=1;    /* 4-byte integer */
  long int  z=2;    /* 8-byte integer   */
};

//down in main
exampleStruct s = {4,5,6};
\end{lstlisting}
\caption{Defining and declaring structs in C\label{fig:CStructs}}
\end{figure}

Compare this to Figure~\ref{fig:AsmStructs}, structures and 
structure variables are both declared in the data segment in assembly. 
%tested in struct1.asm
\begin{figure}[t]
\begin{lstlisting}[language={[x86masm]Assembler}]
.data
   exampleStruct struct
	   x dw 1
	   y dd 2
	   z dq 3
   exampleStruct ends
   s exampleStruct <4,5,6>
\end{lstlisting}
\caption{Defining and declaring structs in Assembly\label{fig:AsmStructs}}
\end{figure}
For both, if we wanted to access member y, we would use {\code s.y}.  If you only wanted
to specify values for x and z you can omit the y initializer and it will be assigned the default 
value: {\code s2 exampleStruct <7,,9>}.  

\subsection{Structure Example}
Here is an example of a structure that holds user data:
\lstinputlisting{../code/ch7/structEx.asm}
%maybe add more about structs, what else do they need to know? 

 %structs
%\include{pcasm8} %floats 
\begin{appendix}
%appendix
\chapter{80x86 Instructions}
\section{Non-floating Point Instructions}
This section lists and describes the actions and formats of the 
non-floating point instructions of the Intel 80x86 CPU family.

The formats use the following abbreviations:
\begin{center}
\begin{tabular}{|l|l|}
\hline
R   & general register \\
R8  & 8-bit register \\
R16 & 16-bit register \\
R32 & 32-bit register \\
SR  & segment register \\
M   & memory \\
M8  & byte \\
M16 & word \\
M32 & double word \\
I   & immediate value \\
\hline
\end{tabular}
\end{center}
These can be combined for the multiple operand instructions. For example,
the format \emph{R, R} means that the instruction takes two register operands.
Many of the two operand instructions allow the same operands. The abbreviation
\emph{O2} is used to represent these operands: \emph{R,R R,M R,I M,R M,I}. If
a 8-bit register or memory can be used for an operand, the abbreviation,
\emph{R/M8} is used.

The table also shows how various bits of the FLAGS register are affected by
each instruction. If the column is blank, the corresponding bit is not
affected at all. If the bit is always changed to a particular value, a 1 or
0 is shown in the column. If the bit is changed to a value that depends on
the operands of the instruction, a \emph{C} is placed in the column. Finally,
if the bit is modified in some undefined way a \emph{?} appears in the
column. Because the only instructions that change the direction flag are 
{\code CLD} and {\code STD}, it is not listed under the FLAGS columns.

\begin{longtable}{||l|p{1.5in}|p{0.75in}|c|c|c|c|c|c||}
\hline \hline
\multicolumn{1}{||c}{} & 
   \multicolumn{1}{c}{} &
   \multicolumn{1}{c}{} &
  \multicolumn{6}{c||}{\textbf{Flags}} \\ \cline{4-9}
\multicolumn{1}{||c}{\textbf{Name}} & 
   \multicolumn{1}{c}{\textbf{Description}} &
   \multicolumn{1}{c}{\textbf{Formats}} &
   \multicolumn{1}{c}{\textbf{O}} &
   \multicolumn{1}{c}{\textbf{S}} &
   \multicolumn{1}{c}{\textbf{Z}} &
   \multicolumn{1}{c}{\textbf{A}} &
   \multicolumn{1}{c}{\textbf{P}} &
   \multicolumn{1}{c||}{\textbf{C}} \\ \hline \endhead
\hline \hline \endfoot
%                                              O   S   Z   A   P   C
{\code ADC} & Add with Carry & O2            & C & C & C & C & C & C \\
{\code ADD} & Add Integers   & O2            & C & C & C & C & C & C \\
{\code AND} & Bitwise AND    & O2            & 0 & C & C & ? & C & 0 \\
{\code BSWAP} & Byte Swap    & R32           &   &   &   &   &   &  \\
{\code CALL} & Call Routine  & R M I         &   &   &   &   &   &   \\
{\code CBW} & Convert Byte to Word &         &   &   &   &   &   & \\
{\code CDQ} & Convert Dword to Qword &       &   &   &   &   &   & \\
{\code CLC} & Clear Carry &                  &   &   &   &   &   & 0 \\
{\code CLD} & Clear Direction Flag &         &   &   &   &   &   & \\
{\code CMC} & Complement Carry &             &   &   &   &   &   & C \\
{\code CMP} & Compare Integers & O2          & C & C & C & C & C & C \\
{\code CMPSB} & Compare Bytes &              & C & C & C & C & C & C \\
{\code CMPSW} & Compare Words &              & C & C & C & C & C & C \\
{\code CMPSD} & Compare Dwords &             & C & C & C & C & C & C \\
{\code CWD} & Convert Word to Dword into DX:AX & &   &   &   &   &   & \\
{\code CWDE} & Convert Word to Dword into EAX & &   &   &   &   &   & \\
{\code DEC} & Decrement Integer & R M        & C & C & C & C & C & \\
{\code DIV} & Unsigned Divide & R M          & ? & ? & ? & ? & ? & ? \\
{\code ENTER} & Make stack frame & I,0       &   &   &   &   &   & \\
{\code IDIV} & Signed Divide & R M           & ? & ? & ? & ? & ? & ? \\
{\code IMUL} & Signed Multiply & R M R16,R/M16 R32,R/M32 R16,I R32,I 
                                       {\small R16,R/M16,I R32,R/M32,I}
                                             & C & ? & ? & ? & ? & C \\
{\code INC} & Increment Integer & R M        & C & C & C & C & C & \\
{\code INT} & Generate Interrupt & I         &   &   &   &   &   & \\
{\code JA } & Jump Above & I                 &   &   &   &   &   & \\
{\code JAE } & Jump Above or Equal & I       &   &   &   &   &   & \\
{\code JB } & Jump Below & I                 &   &   &   &   &   & \\
{\code JBE } & Jump Below or Equal  & I      &   &   &   &   &   & \\
{\code JC } & Jump Carry & I                 &   &   &   &   &   & \\
{\code JCXZ } & Jump if CX = 0 & I           &   &   &   &   &   & \\
{\code JE } & Jump Equal & I                 &   &   &   &   &   & \\
{\code JG } & Jump Greater & I               &   &   &   &   &   & \\
{\code JGE } & Jump Greater or Equal & I     &   &   &   &   &   & \\
{\code JL } & Jump Less & I                  &   &   &   &   &   & \\
{\code JLE } & Jump Less or Equal & I        &   &   &   &   &   & \\
{\code JMP } & Unconditional Jump & R M I    &   &   &   &   &   & \\
{\code JNA } & Jump Not Above & I            &   &   &   &   &   & \\
{\code JNAE } & Jump Not Above or Equal& I   &   &   &   &   &   & \\
{\code JNB } & Jump Not Below & I            &   &   &   &   &   & \\
{\code JNBE } & Jump Not Below or Equal & I  &   &   &   &   &   & \\
{\code JNC } & Jump No Carry & I             &   &   &   &   &   & \\
{\code JNE } & Jump Not Equal & I            &   &   &   &   &   & \\
{\code JNG } & Jump Not Greater & I          &   &   &   &   &   & \\
{\code JNGE } & Jump Not Greater or Equal & I&   &   &   &   &   & \\
{\code JNL } & Jump Not Less & I             &   &   &   &   &   & \\
{\code JNLE } & Jump Not Less or Equal & I   &   &   &   &   &   & \\
{\code JNO } & Jump No Overflow & I          &   &   &   &   &   & \\
{\code JNS } & Jump No Sign & I              &   &   &   &   &   & \\
{\code JNZ } & Jump Not Zero & I             &   &   &   &   &   & \\
{\code JO } & Jump Overflow & I              &   &   &   &   &   & \\
{\code JPE } & Jump Parity Even & I          &   &   &   &   &   & \\
{\code JPO } & Jump Parity Odd & I           &   &   &   &   &   & \\
{\code JS } & Jump Sign & I                  &   &   &   &   &   & \\
{\code JZ } & Jump Zero & I                  &   &   &   &   &   & \\
{\code LAHF} & Load FLAGS into AH &          &   &   &   &   &   & \\
{\code LEA} & Load Effective Address & R32,M &   &   &   &   &   & \\
{\code LEAVE} & Leave Stack Frame &          &   &   &   &   &   & \\
{\code LODSB} & Load Byte &                  &   &   &   &   &   & \\
{\code LODSW} & Load Word &                  &   &   &   &   &   & \\
{\code LODSD} & Load Dword &                 &   &   &   &   &   & \\
{\code LOOP}  & Loop       & I               &   &   &   &   &   & \\
{\code LOOPE/LOOPZ} & Loop If Equal & I     &   &   &   &   &   & \\
{\code LOOPNE/LOOPNZ} & Loop If Not Equal & I  &   &   &   &   &   & \\
{\code MOV} & Move Data & O2 \mbox{SR,R/M16} R/M16,SR
                                             &   &   &   &   &   & \\
{\code MOVSB} & Move Byte &                  &   &   &   &   &   & \\
{\code MOVSW} & Move Word &                  &   &   &   &   &   & \\
{\code MOVSD} & Move Dword &                 &   &   &   &   &   & \\
{\code MOVSX} & Move Signed & R16,R/M8 R32,R/M8 R32,R/M16
                                             &   &   &   &   &   & \\
{\code MOVZX} & Move Unsigned & R16,R/M8 R32,R/M8 R32,R/M16
                                             &   &   &   &   &   & \\
{\code MUL} & Unsigned Multiply & R M        & C & ? & ? & ? & ? & C \\
{\code NEG} & Negate & R M                   & C & C & C & C & C & C \\
{\code NOP} & No Operation &                 &   &   &   &   &   & \\
{\code NOT} & 1's Complement & R M           &   &   &   &   &   & \\
{\code OR} & Bitwise OR    & O2              & 0 & C & C & ? & C & 0 \\
{\code POP} & Pop From Stack & R/M16 R/M32   &   &   &   &   &   & \\
{\code POPA} & Pop All &                     &   &   &   &   &   & \\
{\code POPF} & Pop FLAGS &                   & C & C & C & C & C & C \\
{\code PUSH} & Push to Stack & R/M16 R/M32 I &   &   &   &   &   & \\
{\code PUSHA} & Push All &                   &   &   &   &   &   & \\
{\code PUSHF} & Push FLAGS &                 &   &   &   &   &   & \\
{\code RCL} & Rotate Left with Carry & R/M,I R/M,CL
                                             & C &   &   &   &   & C \\
{\code RCR} & Rotate Right with Carry & R/M,I R/M,CL
                                             & C &   &   &   &   & C \\
{\code REP} & Repeat &                       &   &   &   &   &   & \\
{\code REPE/REPZ} & Repeat If Equal&        &   &   &   &   &   & \\
{\code REPNE/REPNZ} & Repeat If Not Equal&  &   &   &   &   &   & \\
{\code RET} & Return &                       &   &   &   &   &   & \\
{\code ROL} & Rotate Left & R/M,I R/M,CL     & C &   &   &   &   & C \\
{\code ROR} & Rotate Right & R/M,I R/M,CL    & C &   &   &   &   & C \\
{\code SAHF} & Copies AH into FLAGS &        &   & C & C & C & C & C \\
{\code SAL} & Shifts to Left & R/M,I R/M, CL &   &   &   &   &   & C \\
{\code SBB}  & Subtract with Borrow & O2     & C & C & C & C & C & C \\
{\code SCASB} & Scan for Byte &              & C & C & C & C & C & C \\
{\code SCASW} & Scan for Word &              & C & C & C & C & C & C \\
{\code SCASD} & Scan for Dword &             & C & C & C & C & C & C \\
{\code SETA } & Set Above & R/M8                 &   &   &   &   &   & \\
{\code SETAE } & Set Above or Equal & R/M8       &   &   &   &   &   & \\
{\code SETB } & Set Below & R/M8                 &   &   &   &   &   & \\
{\code SETBE } & Set Below or Equal  & R/M8      &   &   &   &   &   & \\
{\code SETC } & Set Carry & R/M8                 &   &   &   &   &   & \\
{\code SETE } & Set Equal & R/M8                 &   &   &   &   &   & \\
{\code SETG } & Set Greater & R/M8               &   &   &   &   &   & \\
{\code SETGE } & Set Greater or Equal & R/M8     &   &   &   &   &   & \\
{\code SETL } & Set Less & R/M8                  &   &   &   &   &   & \\
{\code SETLE } & Set Less or Equal & R/M8        &   &   &   &   &   & \\
{\code SETNA } & Set Not Above & R/M8            &   &   &   &   &   & \\
{\code SETNAE } & Set Not Above or Equal& R/M8   &   &   &   &   &   & \\
{\code SETNB } & Set Not Below & R/M8            &   &   &   &   &   & \\
{\code SETNBE } & Set Not Below or Equal & R/M8  &   &   &   &   &   & \\
{\code SETNC } & Set No Carry & R/M8             &   &   &   &   &   & \\
{\code SETNE } & Set Not Equal & R/M8            &   &   &   &   &   & \\
{\code SETNG } & Set Not Greater & R/M8          &   &   &   &   &   & \\
{\code SETNGE } & Set Not Greater or Equal & R/M8&   &   &   &   &   & \\
{\code SETNL } & Set Not Less & R/M8             &   &   &   &   &   & \\
{\code SETNLE } & Set Not LEss or Equal & R/M8   &   &   &   &   &   & \\
{\code SETNO } & Set No Overflow & R/M8          &   &   &   &   &   & \\
{\code SETNS } & Set No Sign & R/M8              &   &   &   &   &   & \\
{\code SETNZ } & Set Not Zero & R/M8             &   &   &   &   &   & \\
{\code SETO } & Set Overflow & R/M8              &   &   &   &   &   & \\
{\code SETPE } & Set Parity Even & R/M8          &   &   &   &   &   & \\
{\code SETPO } & Set Parity Odd & R/M8           &   &   &   &   &   & \\
{\code SETS } & Set Sign & R/M8                  &   &   &   &   &   & \\
{\code SETZ } & Set Zero & R/M8                  &   &   &   &   &   & \\

{\code SAR} & Arithmetic Shift to Right & R/M,I R/M, CL 
                                             &   &   &   &   &   & C \\
{\code SHR} & Logical Shift to Right & R/M,I R/M, CL 
                                             &   &   &   &   &   & C \\
{\code SHL} & Logical Shift to Left & R/M,I R/M, CL 
                                             &   &   &   &   &   & C \\
{\code STC} & Set Carry &                    &   &   &   &   &   & 1 \\
{\code STD} & Set Direction Flag &           &   &   &   &   &   & \\
{\code STOSB} & Store Btye &                 &   &   &   &   &   & \\
{\code STOSW} & Store Word &                 &   &   &   &   &   & \\
{\code STOSD} & Store Dword &                &   &   &   &   &   & \\
{\code SUB} & Subtract & O2                  & C & C & C & C & C & C\\
{\code TEST} & Logical Compare & R/M,R R/M,I & 0 & C & C & ? & C & 0\\
{\code XCHG} & Exchange & R/M,R R,R/M        &   &   &   &   &   & \\
{\code XOR} & Bitwise XOR    & O2            & 0 & C & C & ? & C & 0 \\

\end{longtable}

\newpage
\section{Floating Point Instructions}

\renewcommand{\thefootnote}{\fnsymbol{footnote}} In this section, many
of the 80x86 math coprocessor instructions are described. The
description section briefly describes the operation of the
instruction. To save space, information about whether the instruction
pops the stack is not given in the description. 

The format column shows what type of operands can be used with each
instruction. The following abbreviations are used:
\begin{center}
\begin{tabular}{|l|l|}
\hline
Abbr.      & Meaning\\
\hline
ST\emph{n} & A coprocessor register \\
F          & Single precision number in memory \\
D          & Double precision number in memory \\
E          & Extended precision number in memory \\
I16        & Integer word in memory \\
I32        & Integer double word in memory \\
I64        & Integer quad word in memory \\
\hline
\end{tabular}
\end{center}

Instructions requiring a Pentium Pro or better are marked with an 
asterisk(\footnotemark[1]).

\begin{longtable}{||l|l|l||}
\hline \hline
\textbf{Instruction} &  \textbf{Description} & \textbf{Format} \\
\hline
\endhead
\hline \hline \endfoot
{\code FABS} & $\mathtt{ST0} = |\mathtt{ST0}|$ & \\
{\code FADD \emph{src}} & {\code ST0 += \emph{src}} & ST\emph{n} F D \\
{\code FADD \emph{dest}, ST0} & {\code \emph{dest} += STO} & ST\emph{n} \\
{\code FADDP \emph{dest}[,ST0]} & {\code \emph{dest} += ST0} & ST\emph{n} \\
{\code FCHS} & $\mathtt{ST0} = - \mathtt{ST0}$ & \\
{\code FCOM \emph{src}} & Compare {\code ST0} and {\code \emph{src}} &
ST\emph{n} F D \\
{\code FCOMP \emph{src}} & Compare {\code ST0} and {\code \emph{src}} &
ST\emph{n} F D \\
{\code FCOMPP \emph{src}} & Compares {\code ST0} and {\code ST1} & \\
{\code FCOMI\footnotemark[1] \emph{src}} & Compares into FLAGS 
& ST\emph{n} \\
{\code FCOMIP\footnotemark[1] \emph{src}} & Compares into FLAGS 
& ST\emph{n} \\
{\code FDIV \emph{src}} & {\code ST0 /= \emph{src}} & ST\emph{n} F D \\
{\code FDIV \emph{dest}, ST0} & {\code \emph{dest} /= STO} & ST\emph{n} \\
{\code FDIVP \emph{dest}[,ST0]} & {\code \emph{dest} /= ST0} & ST\emph{n} \\
{\code FDIVR \emph{src}} & {\code ST0 = \emph{src}/ST0} & ST\emph{n} F D \\
{\code FDIVR \emph{dest}, ST0} & {\code \emph{dest} = ST0/\emph{dest}} 
& ST\emph{n} \\
{\code FDIVRP \emph{dest}[,ST0]} & {\code \emph{dest} = ST0/\emph{dest}} 
& ST\emph{n} \\
{\code FFREE \emph{dest}} & Marks as empty & ST\emph{n} \\
{\code FIADD \emph{src}} & {\code ST0 += \emph{src}} & I16 I32 \\
{\code FICOM \emph{src}} & Compare {\code ST0} and {\code \emph{src}} &
I16 I32 \\
{\code FICOMP \emph{src}} & Compare {\code ST0} and {\code \emph{src}} &
I16 I32 \\
{\code FIDIV \emph{src}} & {\code STO /= \emph{src}} & I16 I32 \\
{\code FIDIVR \emph{src}} & {\code STO = \emph{src}/ST0} & I16 I32 \\
{\code FILD \emph{src}} & Push \emph{src} on Stack & I16 I32 I64 \\
{\code FIMUL \emph{src}} & {\code ST0 *= \emph{src}} & I16 I32 \\
{\code FINIT} & Initialize Coprocessor & \\
{\code FIST \emph{dest}} & Store {\code ST0} & I16 I32 \\
{\code FISTP \emph{dest}} & Store {\code ST0} & I16 I32 I64\\
{\code FISUB \emph{src}} & {\code ST0 -= \emph{src}} & I16 I32 \\
{\code FISUBR \emph{src}} & {\code ST0 = \emph{src} - ST0} & I16 I32 \\
{\code FLD \emph{src}} & Push \emph{src} on Stack & ST\emph{n} F D E \\
{\code FLD1} & Push 1.0 on Stack & \\
{\code FLDCW \emph{src}} & Load Control Word Register & I16 \\
{\code FLDPI} & Push $\pi$ on Stack & \\
{\code FLDZ} & Push 0.0 on Stack & \\
{\code FMUL \emph{src}} & {\code ST0 *= \emph{src}} & ST\emph{n} F D \\
{\code FMUL \emph{dest}, ST0} & {\code \emph{dest} *= STO} & ST\emph{n} \\
{\code FMULP \emph{dest}[,ST0]} & {\code \emph{dest} *= ST0} & ST\emph{n} \\
{\code FRNDINT} & Round {\code ST0} & \\
{\code FSCALE} & $\mathtt{ST0} = \mathtt{ST0} \times 2^{\lfloor \mathtt{ST1} \rfloor}$ & \\
{\code FSQRT} & $\mathtt{ST0} = \sqrt{\mathtt{STO}}$ & \\
{\code FST \emph{dest}} & Store {\code ST0} & ST\emph{n} F D \\
{\code FSTP \emph{dest}} & Store {\code ST0} & ST\emph{n} F D E \\
{\code FSTCW \emph{dest}} & Store Control Word Register & I16 \\
{\code FSTSW \emph{dest}} & Store Status Word Register & I16 AX \\
{\code FSUB \emph{src}} & {\code ST0 -= \emph{src}} & ST\emph{n} F D \\
{\code FSUB \emph{dest}, ST0} & {\code \emph{dest} -= STO} & ST\emph{n} \\
{\code FSUBP \emph{dest}[,ST0]} & {\code \emph{dest} -= ST0} & ST\emph{n} \\
{\code FSUBR \emph{src}} & {\code ST0 = \emph{src}-ST0} & ST\emph{n} F D \\
{\code FSUBR \emph{dest}, ST0} & {\code \emph{dest} = ST0-\emph{dest}} 
& ST\emph{n} \\
{\code FSUBP \emph{dest}[,ST0]} & {\code \emph{dest} = ST0-\emph{dest}} 
& ST\emph{n} \\
{\code FTST} & Compare {\code ST0} with 0.0 & \\
{\code FXCH \emph{dest}} & Exchange {\code ST0} and {\code \emph{dest}} 
& ST\emph{n} \\
\end{longtable}

\renewcommand{\thefootnote}{\arabic{footnote}}

\newpage
\chapter{Interrupts\index{interrupt}}
\section{Common Interrupts}
We can do a wide variety of useful system interactions using interrupts.  There are too many to list, but below are some interesting examples (see Table ~\ref{fig:interrupts} \footnote{table originally from \url{http://www.skynet.ie/~darkstar/assembler/intlist.html}}
).  \\
\indent The format for the interrupts follow the same pattern.  You load the values into the appropriate registers and then call the interrupt.  Here is an example to output one character:
\begin{lstlisting}[language={[x86masm]Assembler}]
    ;print @ char
    mov ah,2     ; setting for outputting a char
    mov dl,'@'   ; dl = character going out  
    int 21h      ; call the interrupt and output the character
\end{lstlisting}
You can output one character at a time or a whole string.  You can also read a character from the keyboard, here is an example of both working together:
\lstinputlisting{../code/readkey.asm}
\begin{figure}[h]
%\begin{table}[]
\hskip-1.5cm% shifts table left
\begin{tabular}{| p{4cm} | p{4.5cm} | p{4cm} | p{3.5cm} |}
%\begin{tabular}{| l | l | l | l |}
\hline
\textbf{Interrupt}                & \textbf{AH= (SubFunction)}       & \textbf{Input}            & \textbf{Output}         \\
\hline
10h 					 & 00 (SET\_MODE)    &AL=mode number     & -                                \\
(VIDEO\_INTERRUPT)      & Sets the Video mode.      &                                  &       			        \\
\cline{2-4} 
                                           & 0Ch (WRITE\_DOT)        & DX=row  &\\
                                           & Puts a dot on the screen & CX=column   & - \\
                                           & Graphics modes only      &  AL=color   &\\                                                                                                                     \cline{2-4}     			 
                                           & 0Dh (READ\_DOT)         & DX=row        	    & AL=color     \\                                                                            
                                           & Reads a dot on screen   & CX=column & \\
                                           & Graphics modes only      & &\\
\hline                                        
16h 		               & 00 (AWAIT\_CHAR)  		       & -       & AL=character in \\
(KBD\_IO) 	       & Reads a character from keyboard &		  & AH=scancode \\
\cline{2-4} 		       
			       & 01 (PREVIEW\_KEY) 		      &          & Zero flag = key ready \\ 
			       & Checks to see if a key is      & - 	 & AL=character in \\
			       & ready, but does not  &		 & AH=scancode \\
			       & remove key from buffer. & &\\
\hline
21h   			 & 01 (KEYBOARD\_INPUT)               & -       & AL=character in \\      
(DOS\_INTERRUPT) & Reads and displays one character   &         &           \\
\cline{2-4} 
		                      & 02 (DISPLAY\_OUTPUT)                    & DL=character out       & -       \\                                                                                                                                            				     & Displays one character on screen      &  & \\
\cline{2-4} 
				    & 08 (NO\_ECHO\_INPUT)              & -                             & AL=character in   \\
				    & Same as 01 but not displayed       & &\\       
\cline{2-4}     
			 	   & 09 (PRINT\_STRING)           	   & DX=address of output           &         \\
				   & Displays a string on screen            & string & - \\
				   & String must end with "\$"		   &   &  \\
%\cline{2-4}        
%                  		   & 0A (BUFFERED\_INPUT)              & DX=address of buffer  & Second char of buffer=length of input \\  
%		   		   & Reads a string from keyboard        & First character=max length & Rest of buffer=input string \\
%		   		   &							  &						& followed by carriage return (0Dh)\\
\cline{2-4}    
		                    & 4Ch (EXIT)                                    & AL=exit code                         & -      \\
\hline                                                                                                                                  
\end{tabular}
%\end{table}
\caption{Common Interrupts \label{fig:interrupts}}
\end{figure}
%\footnote{\href {http://www.skynet.ie/~darkstar/assembler/intlist.html}{http://www.skynet.ie/~darkstar/assembler/intlist.html}}. 
\end{appendix}
\clearpage
\ifmypdf
\phantomsection % fixes link anchor
\fi
\addcontentsline{toc}{chapter}{Index}
\printindex
\end{document}

